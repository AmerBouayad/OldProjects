\documentclass[french,10pt,A4]{report}
%\usepackage[latin1]{inputenc}
\usepackage[utf8]{inputenc}
\usepackage[francais]{babel}
% Modification des marges ------------------------------
\oddsidemargin -4mm % Decreases the left margin by 4mm
\textwidth 17cm %Sets text width across page = 17cm
\textheight 22cm %Sets text height up and down = 22cm
% -----------------------------------------------------
\usepackage[T1]{fontenc}
\usepackage{graphicx,color, caption2}
\usepackage{epsfig}
\usepackage{fancyhdr}
\pagestyle{fancy}
\usepackage{listings}
\usepackage{color}
\usepackage{makeidx}
\usepackage{latexsym}

% Valeurs par défaut le lstset
\lstset{language={},%C,Assembleur, TeX, tcl, basic, cobol, fortran, logo, make, pascal, perl, prolog, {}
	literate={â}{{\^a}}1 {ê}{{\^e}}1 {î}{{\^i}}1 {ô}{{\^o}}1 {û}{{\^u}}1
		 {ä}{{\"a}}1 {ë}{{\"e}}1 {ï}{{\"i}}1 {ö}{{\"o}}1 {ü}{{\"u}}1
		 {à}{{\`a}}1 {é}{{\'e}}1 {è}{{\`e}}1 {ù}{{\`u}}1 
		 {Â}{{\^A}}1 {Ê}{{\^E}}1 {Î}{{\^I}}1 {Ô}{{\^O}}1 {Û}{{\^U}}1
		 {Ä}{{\"A}}1 {Ë}{{\"E}}1 {Ï}{{\"I}}1 {Ö}{{\"O}}1 {Ü}{{\"U}}1
		 {À}{{\`A}}1 {É}{{\'E}}1 {È}{{\`E}}1 {Ù}{{\`U}}1,
	commentstyle=\scriptsize\ttfamily\slshape, % style des commentaires
	basicstyle=\scriptsize\ttfamily, % style par défaut
	keywordstyle=\scriptsize\rmfamily\bfseries,% style des mots-clés
	backgroundcolor=\color[rgb]{.95,.95,.95}, % couleur de fond : gris clair
	framerule=0.5pt,% Taille des bords
	frame=trbl,% Style du cadre
	frameround=tttt, % Bords arrondis 
	tabsize=3, % Taille des tabulations
%	extendedchars=\true, % Incompatible avec utf8 et literate
	inputencoding=utf8,
	showspaces=false, % Ne montre pas les espaces 
	showstringspaces=false, % Ne montre pas les espaces entre ''
	xrightmargin=-1cm, % Retrait gauche 
	xleftmargin=-1cm, % Retrait droit
	escapechar=°}  % Caractère d'échappement, permet des commandes latex dans la source
% -----------------------------------------------------
%\makeindex
\begin{document}
\lhead{Labo Introduction S.E 2$^eme$}
\rhead{Page \thepage}
\lfoot{\copyright MBA }
\rfoot{\today}
\cfoot{ }
\renewcommand{\footrulewidth}{0.4pt}

\setlength{\parindent}{0pt} % pas d'indentation

\lstset{frame=trBL}

\setcounter{tocdepth}{1}	% limiter les nivaux de table des matières
\setcounter{secnumdepth}{5}	% La numérotation des sections au maximum

\newcommand{\titre}{Titre du sujet}	% la variable contenant le titre du sujet

\thispagestyle{empty}

\title{\emph{Laboratoire\\\textbf{Introduction}}}
\author{mba, mwa, nvs}
\date{septembre 2018}
\maketitle
\tableofcontents
% ~\\[5cm]
% \flushright{Dis, papa, ça veut dire quoi "Formating drive c" ?}
% \flushleft{ }
\newpage
%
\chapter{Makefile simple}
	\lstset{language=c}
\renewcommand{\titre}{\textcolor{blue}{ Intro - LaboIntro 01-01 - un Makefile }}

\lhead{ \titre }
\section{{\titre} }

\begin{tabular}{|l|l|}
\hline
Titre : 	& \titre \\\hline
Support : 	& OS 43.2 Leap\\\hline
Date :		& 08/2018 \\\hline
\end{tabular}

\subsection{Make}
L'utilitaire \emph{make} permet de déterminer quelles parties d'un programme complexe nécessitent une compilation et exécute les commandes qui lui auront été renseignées pour compiler. Nous allons l'utiliser dans ce cadre.

Un Makefile est un fichier contenant des règles répondant au format suivant :\\

%\lstinputlisting{LaboIntro0101/regle}
\emph{cible : pré-requis ...}\\
\emph{<tab> commande }\\


\emph{make} sait ainsi que pour produire \emph{cible} il aura besoin de \emph{pré-requis} et devra exécuter \emph{commande}

\subsection{Énoncé}

Écrivez un Makefile contenant plusieurs règles.

La première (qui est la règle par défaut) exécute votre programme \emph{TestMake}. Ce petit exécutable ne fait rien de spécial, il affiche trois représentations d'une même  valeur. \\ Le fichier exécutable TestMake doit être généré si il n'existe pas ou n'est pas à jour. Vous allez avoir besoin d'une règle pour décrire cela. Comme contrainte, le code TestMake.c utilise un fichier d'include \emph{MonInclude.h}.

Vous écrirez également la règle ou cible \emph{clean} qui n'a aucun pré-requis et sert juste à supprimer l'exécutable \emph{TestMake} pour revenir à une situation de départ ou "faire du propre", comme son nom l'indique.

Exécutez ensuite manuellement la commande \texttt{make clean} suivie de deux fois la commande \texttt{make}. 

Observez et justifiez la différence de comportement entre ces deux dernières exécutions. 

Modifiez le fichier .c ou/et le fichier .h (ou modifiez leur date par la commande \emph{touch <fichier>}).

Quel comportement génère la commande \texttt{make} maintenant ?

\newpage % ne sait pas comment empècher les sautx page au milieu du code ...
\subsection{Une solution}

Voir dans LaboIntro0101/SOURCES

\lstinputlisting{LaboIntro0101/SOURCES/Makefile}
\lstinputlisting{LaboIntro0101/SOURCES/MonInclude.h}
\lstinputlisting{LaboIntro0101/SOURCES/TestMake.c}

\subsection{Commentaires}

\begin{itemize}
\item nous choisissons le nom Makefile pour le fichier makefile  c'est un des noms de fichier que l'outil \texttt{make} emploie par défaut.
\item un programme ne doit pas être recompilé à chaque exécution.
\item le programme nécessite d'être recompilé si le fichier .c ou le fichier .h dont il dépend, ont une date plus récente que l'exécutable. Ceci est exprimé par une règle et sera interprété par l'outil \texttt{make}.
\item un pré-requis dans une règle de Makefile n'est pas toujours un nom de fichier il peut s'agir d'un nom de règle.
\end{itemize}

\subsection{Voir}
Pour une description complète de l'outil \texttt{make}

GNU Make : (Stallman, McGrath, Smith)
\newpage
 % Makefile
\chapter{Un Makefile à améliorer}
	\lstset{language=c}
\renewcommand{\titre}{\textcolor{blue}{ Intro - LaboIntro 01-02 - Compilation de plusieurs sources }}

\lhead{ \titre }
\section{{\titre} }

\begin{tabular}{|l|l|}
\hline
Titre : 	& \titre \\\hline
Support : 	& 0S 42.3 Leap \\\hline
Date :		& 08/2016\\\hline
\end{tabular}

\subsection{Énoncé}
Écrivez un Makefile qui réalise la compilation et l'édition de liens de plusieurs sources en C.

\lstinputlisting{LaboIntro0102/SOURCES/Makefile}

\lstinputlisting{LaboIntro0102/SOURCES/MonInclude.h}

\lstinputlisting{LaboIntro0102/SOURCES/TestMake2.c}

\lstinputlisting{LaboIntro0102/SOURCES/boucle.h}

\lstinputlisting{LaboIntro0102/SOURCES/boucle.c}

\subsection{Commentaires}

Le fichier Makefile de cet exercice n'est pas écrit de manière optimale.

En effet la moindre modification de fichier source c ou de fichier d'include amène à recompiler chaque fichier source et refaire l'édition de liens des objets générés pour produire le fichier exécutable. 
\\Or, lorsque un fichier source est modifié, seul ce dernier nécessite d'être recompilé avant de refaire l'édition de liens. \\
La compilation de chaque source n'est donc nécessaire que si le fichier source même ou un des fichiers d'include qu'il utilise ont été modifiés. \\
Par ailleurs, une édition de liens pour générer un exécutable est nécessaire dès qu'un fichier objet qui le construit a été généré par une compilation.\\ Nous venons d'établir des dépendances entre fichiers : un exécutable dépend de fichiers objet qui à leur tour dépendent de fichiers sources et include. Ceci va nous aider à améliorer notre Makefile. 

\begin{itemize}
\item Il est nécessaire de recompiler une unité de compilation (et donc générer un fichier objet .o) si et seulement si le fichier .o a été supprimé ou le fichier source ou un des include files a été modifié.
\item Il est nécessaire de faire une édition de liens si et seulement si le fichier exécutable a été supprimé ou un des fichiers .o (objets) est plus récent que l'exécutable. Ces derniers nécessitent d'ailleurs peut-être d'être rafraîchis en premier lieu. Les pré-requis sont examinés en cascade dans le cas de dépendances.

\end{itemize}

\subsection{En Roue Libre}
\par Identifiez les dépendances réelles entre les différents fichiers et réécrivez les règles en respectant ces dernières de manière à ce que le minimum de compilations indispensables soit réalisé (une des deux compilations et où l'édition de liens).

\par Une compilation séparée est obtenue via l'option -c du compilateur.\\

\par Réécrivez un Makefile amélioré et vérifiez les
 point suivants :
\par la commande  \texttt{make} ne réalise dès à présent l'édition de liens que lorsque elle est précédée par la commande \texttt{touch boucle.o}.

\par la commande  \texttt{make} ne recompile pas le fichier boucle.c lorsqu'elle est précédée par la commande \texttt{touch MonInclude.h}.
\newpage
 % Makefile
\chapter{Automatiser la démonstration}
	\lstset{language=c}
\renewcommand{\titre}{\textcolor{blue}{ Intro - LaboIntro 01-03 - script Demo}}

\lhead{ \titre }
\section{{\titre} }

\begin{tabular}{|l|l|}
\hline
Titre : 	& \titre \\\hline
Support : 	& 0S 42.3 Leap\\\hline
Date :		& 08/2016 \\\hline
\end{tabular}

\subsection{Énoncé}
Il est temps de s'attarder sur l'écriture d'un script de démonstration bash, on le nommera \emph{Demo}.\\
Soit un Makefile solution de l'exercice précédent que l'on renommera ici MonMakefile, essayons de montrer de manière automatisée que le comportement de MonMakefile correspond à notre attente. 

Vous veillerez à utiliser la commande \texttt{chmod} pour donner le droit d'exécution au script Demo.\\
Vous pourrez dès lors lancer son exécution par la commande \texttt{./Demo}\\
La première ligne du script est \texttt {\#!/bin/bash}. Lorsque on essaye d'exécuter ce script via la commande ./Demo dans le shell, cette ligne indique au shell la nature de ce fichier : un fichier de commandes destiné à l'interpréteur \texttt{/bin/bash} et non un exécutable.\\

\subsection{Une Solution}

\lstinputlisting{LaboIntro0103/SOURCES/Demo}

\subsection{Commentaires}

\begin{itemize}
\item Le script créé ici a pour seul but de montrer de manière automatisée le comportement de notre fichier MonMakefile. 
\item L'écriture de scripts est très intéressante dans le cas de tâches administratives récurrentes.
\item Lors de nos laboratoires de SYStème, nous allons travailler un peu différemment : le script sera la démonstration de l'exercice. L'exercice  qui suit celui-ci est une illustration de cette manière de travailler.
\item Tous nos exercices de laboratoire utiliseront ces deux outils. 
\end{itemize}
\subsection{Voir}
man bash
Scripts sous linux - C. Blaess
\newpage
 % Makefile
\chapter{Un premier exercice : découvrir le shell}
	\lstset{language=c}
\renewcommand{\titre}{\textcolor{blue}{ Intro - LaboIntro 01-04 - Fonctionnement d'un shell, premiers pas}}

\lhead{ \titre }
\section{{\titre} }

\begin{tabular}{|l|l|}
\hline
Titre : 	& \titre \\\hline
Support : 	& OS 42.3 Leap \\\hline
Date :		& 08/2016 \\\hline
\end{tabular}

\subsection{Énoncé}

Nous allons nous servir de la commande \texttt{echo} pour comprendre comment le shell traite une ligne de commande et les wildcards.

\texttt{echo} est une commande qui affiche sur la sortie standard ce qu'elle a reçu en paramètre.
Observez le comportement de la commande \texttt{echo} dans les cas suivants : \texttt{echo coucou ; echo une phrase complète > fichier ; echo coucou >{}> fichier ; echo * ; echo coucou | wc -c} 

Dans le sous-répertoire SOURCES, Écrivez un code source en c et baptisez le Mecho.c. Il affichera sur la sortie standard ce qu'il a reçu en paramètre.

Cette commande se comporte-t-elle comme la commande \texttt{echo} dans les cas cités précédemment ?


\subsection{Une solution}

Pour résoudre cet exercice il faut écrire le code source Mecho.c, le script Demo qui exécute les différents tests demandés et le Makefile qui génère l'exécutable et lance la démonstration. Ce dernier aura également une règle \emph{clean} appropriée.

Ainsi \texttt { cd SOURCES ; make} montre l'exercice résolu. 

Les fichiers suivants se trouvent dans le répertoire SOURCES :

\lstinputlisting{LaboIntro0104/SOURCES/Mecho.c}
\lstinputlisting{LaboIntro0104/SOURCES/Demo}
\lstinputlisting{LaboIntro0104/SOURCES/Makefile}

\subsection{En roue libre : complétons Demo}
\begin{itemize}
\item est-ce que, pour les cas suivants, le programme Mecho se comportera comme \texttt{echo} ?
\begin{itemize}
\item \texttt{echo une phrase complète > fichier}
\item \texttt{echo coucou >{}> fichier}
\item \texttt{echo coucou | wc -c}
\item \texttt{echo *}
\end{itemize}
\item complétez votre script Demo pour vérifier ces derniers cas
\end{itemize}

\subsection{Commentaires}

\begin{itemize}
\item argv[0] n'est pas affiché. En effet, en C, le premier argument est toujours le nom du programme, dans ce cas Mecho.
\item Remarquons que lorsque on utilise les wildcards (\texttt{*, ?, ...}), les redirections (\texttt{>,>{}>}) et les pipes (\texttt{|}), ceux-ci ne sont pas affichés par Mecho. Mecho tout comme \texttt{echo} ne reçoit pas ces symboles en paramètre. Ils sont en effet filtrés et traités par le Shell. 
\item En écrivant notre commande Mecho il apparaît clairement que c'est le shell qui interprète ces caractères spéciaux et réalise les opérations qu'ils cachent, avant d'exécuter votre programme. De la même manière, dans la commande \texttt{ls > f} ce n'est pas la commande \texttt{ls} qui gère la redirection.
\item Quand vous écrirez votre propre shell (bientôt) vous aurez appris à gérer les redirections les pipes et les wildcards tout comme le shell le fait ...
\item Le but de cet exercice était de découvrir une partie de travail que le shell prend en charge lorsque une commande est appelée. Il est donc très important de mentionner cela dans ce commentaire ;-)
\item Vous travaillerez de cette manière pour tous les exercices du laboratoire.
\end{itemize}
\subsection{Voir}
\texttt{man bash} pour le fonctionnement du shell bash
\newpage
 % Script
\chapter{Makefile, règles implicites}
	\lstset{language=c}
\renewcommand{\titre}{\textcolor{blue}{ Intro - LaboIntro 01-05 - sans Makefile }}

\lhead{ \titre }
\section{{\titre} }

\begin{tabular}{|l|l|}
\hline
Titre : 	& \titre \\\hline
Support : 	& OS 42.3 Leap \\\hline
Date :		& 01/2016 \\\hline
\end{tabular}

\subsection{Énoncé}

Reprenez les fichiers source de la solution de l'exercice 0101, on renommera TestMake.c TestMake2.c. 
Ajoutez un deuxième affichage d'une constante définie dans un deuxième fichier d'include de nom TestMake2.h (même préfixe que le code c)

Exécutez la commande \texttt{make TestMake2} sans avoir créé de fichier Makefile.

Quel comportement a la commande dans ce cas ? 

Répétez la commande après avoir exécuté la commande \texttt{touch TestMake2.c}

Répétez la commande après avoir modifié la valeur de la constante MAX dans le fichier \emph{MonInclude.h}

Répétez la commande après avoir modifié la valeur de la constante MIN dans le fichier \emph{TestMake2.h}

\lstinputlisting{LaboIntro0105/SOURCES/MonInclude.h}
\lstinputlisting{LaboIntro0105/SOURCES/TestMake2.h}
\lstinputlisting{LaboIntro0105/SOURCES/TestMake2.c}

\subsection{Commentaires}

\begin{itemize}
\item \texttt{make} utilise ici des règles par défaut pour les compilations en c.
\item \texttt{make} considère à défaut d'information qu'une cible dépendra de fichiers de même préfixe. Dans ce cas on doit générer TestMake2 et le fichier TestMake2.c est présent.
\item Dans ce cas \texttt{make} fait appel au compilateur cc
\item Pour l'exemple nous avons ajouté un fichier d'include de même préfixe que le fichier exécutable \emph{TestMake2}.
\item Toutefois, les liens avec des éventuels fichiers d'include ne sont pas établis par défaut, même si ceux-ci ont le même préfixe. Une modification de \emph{TestMake2.h} ne génère  pas de nouvelle compilation.
\item En présence d'un Makefile, \texttt{make} essayera de générer de cette même manière, une cible pour laquelle aucune règle explicite n'est fournie.
\item Nous préférerons l'écriture de règles explicites
\end{itemize}
\newpage
 % Script
\end{document}
