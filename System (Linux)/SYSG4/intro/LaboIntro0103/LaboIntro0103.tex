\lstset{language=c}
\renewcommand{\titre}{\textcolor{blue}{ Intro - LaboIntro 01-03 - script Demo}}

\lhead{ \titre }
\section{{\titre} }

\begin{tabular}{|l|l|}
\hline
Titre : 	& \titre \\\hline
Support : 	& 0S 42.3 Leap\\\hline
Date :		& 08/2016 \\\hline
\end{tabular}

\subsection{Énoncé}
Il est temps de s'attarder sur l'écriture d'un script de démonstration bash, on le nommera \emph{Demo}.\\
Soit un Makefile solution de l'exercice précédent que l'on renommera ici MonMakefile, essayons de montrer de manière automatisée que le comportement de MonMakefile correspond à notre attente. 

Vous veillerez à utiliser la commande \texttt{chmod} pour donner le droit d'exécution au script Demo.\\
Vous pourrez dès lors lancer son exécution par la commande \texttt{./Demo}\\
La première ligne du script est \texttt {\#!/bin/bash}. Lorsque on essaye d'exécuter ce script via la commande ./Demo dans le shell, cette ligne indique au shell la nature de ce fichier : un fichier de commandes destiné à l'interpréteur \texttt{/bin/bash} et non un exécutable.\\

\subsection{Une Solution}

\lstinputlisting{LaboIntro0103/SOURCES/Demo}

\subsection{Commentaires}

\begin{itemize}
\item Le script créé ici a pour seul but de montrer de manière automatisée le comportement de notre fichier MonMakefile. 
\item L'écriture de scripts est très intéressante dans le cas de tâches administratives récurrentes.
\item Lors de nos laboratoires de SYStème, nous allons travailler un peu différemment : le script sera la démonstration de l'exercice. L'exercice  qui suit celui-ci est une illustration de cette manière de travailler.
\item Tous nos exercices de laboratoire utiliseront ces deux outils. 
\end{itemize}
\subsection{Voir}
man bash
Scripts sous linux - C. Blaess
\newpage
