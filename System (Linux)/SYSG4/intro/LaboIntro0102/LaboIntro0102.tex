\lstset{language=c}
\renewcommand{\titre}{\textcolor{blue}{ Intro - LaboIntro 01-02 - Compilation de plusieurs sources }}

\lhead{ \titre }
\section{{\titre} }

\begin{tabular}{|l|l|}
\hline
Titre : 	& \titre \\\hline
Support : 	& 0S 42.3 Leap \\\hline
Date :		& 08/2016\\\hline
\end{tabular}

\subsection{Énoncé}
Écrivez un Makefile qui réalise la compilation et l'édition de liens de plusieurs sources en C.

\lstinputlisting{LaboIntro0102/SOURCES/Makefile}

\lstinputlisting{LaboIntro0102/SOURCES/MonInclude.h}

\lstinputlisting{LaboIntro0102/SOURCES/TestMake2.c}

\lstinputlisting{LaboIntro0102/SOURCES/boucle.h}

\lstinputlisting{LaboIntro0102/SOURCES/boucle.c}

\subsection{Commentaires}

Le fichier Makefile de cet exercice n'est pas écrit de manière optimale.

En effet la moindre modification de fichier source c ou de fichier d'include amène à recompiler chaque fichier source et refaire l'édition de liens des objets générés pour produire le fichier exécutable. 
\\Or, lorsque un fichier source est modifié, seul ce dernier nécessite d'être recompilé avant de refaire l'édition de liens. \\
La compilation de chaque source n'est donc nécessaire que si le fichier source même ou un des fichiers d'include qu'il utilise ont été modifiés. \\
Par ailleurs, une édition de liens pour générer un exécutable est nécessaire dès qu'un fichier objet qui le construit a été généré par une compilation.\\ Nous venons d'établir des dépendances entre fichiers : un exécutable dépend de fichiers objet qui à leur tour dépendent de fichiers sources et include. Ceci va nous aider à améliorer notre Makefile. 

\begin{itemize}
\item Il est nécessaire de recompiler une unité de compilation (et donc générer un fichier objet .o) si et seulement si le fichier .o a été supprimé ou le fichier source ou un des include files a été modifié.
\item Il est nécessaire de faire une édition de liens si et seulement si le fichier exécutable a été supprimé ou un des fichiers .o (objets) est plus récent que l'exécutable. Ces derniers nécessitent d'ailleurs peut-être d'être rafraîchis en premier lieu. Les pré-requis sont examinés en cascade dans le cas de dépendances.

\end{itemize}

\subsection{En Roue Libre}
\par Identifiez les dépendances réelles entre les différents fichiers et réécrivez les règles en respectant ces dernières de manière à ce que le minimum de compilations indispensables soit réalisé (une des deux compilations et où l'édition de liens).

\par Une compilation séparée est obtenue via l'option -c du compilateur.\\

\par Réécrivez un Makefile amélioré et vérifiez les
 point suivants :
\par la commande  \texttt{make} ne réalise dès à présent l'édition de liens que lorsque elle est précédée par la commande \texttt{touch boucle.o}.

\par la commande  \texttt{make} ne recompile pas le fichier boucle.c lorsqu'elle est précédée par la commande \texttt{touch MonInclude.h}.
\newpage
