\lstset{language=c}
\renewcommand{\titre}{\textcolor{blue}{ Intro - LaboIntro 01-05 - sans Makefile }}

\lhead{ \titre }
\section{{\titre} }

\begin{tabular}{|l|l|}
\hline
Titre : 	& \titre \\\hline
Support : 	& OS 42.3 Leap \\\hline
Date :		& 01/2016 \\\hline
\end{tabular}

\subsection{Énoncé}

Reprenez les fichiers source de la solution de l'exercice 0101, on renommera TestMake.c TestMake2.c. 
Ajoutez un deuxième affichage d'une constante définie dans un deuxième fichier d'include de nom TestMake2.h (même préfixe que le code c)

Exécutez la commande \texttt{make TestMake2} sans avoir créé de fichier Makefile.

Quel comportement a la commande dans ce cas ? 

Répétez la commande après avoir exécuté la commande \texttt{touch TestMake2.c}

Répétez la commande après avoir modifié la valeur de la constante MAX dans le fichier \emph{MonInclude.h}

Répétez la commande après avoir modifié la valeur de la constante MIN dans le fichier \emph{TestMake2.h}

\lstinputlisting{LaboIntro0105/SOURCES/MonInclude.h}
\lstinputlisting{LaboIntro0105/SOURCES/TestMake2.h}
\lstinputlisting{LaboIntro0105/SOURCES/TestMake2.c}

\subsection{Commentaires}

\begin{itemize}
\item \texttt{make} utilise ici des règles par défaut pour les compilations en c.
\item \texttt{make} considère à défaut d'information qu'une cible dépendra de fichiers de même préfixe. Dans ce cas on doit générer TestMake2 et le fichier TestMake2.c est présent.
\item Dans ce cas \texttt{make} fait appel au compilateur cc
\item Pour l'exemple nous avons ajouté un fichier d'include de même préfixe que le fichier exécutable \emph{TestMake2}.
\item Toutefois, les liens avec des éventuels fichiers d'include ne sont pas établis par défaut, même si ceux-ci ont le même préfixe. Une modification de \emph{TestMake2.h} ne génère  pas de nouvelle compilation.
\item En présence d'un Makefile, \texttt{make} essayera de générer de cette même manière, une cible pour laquelle aucune règle explicite n'est fournie.
\item Nous préférerons l'écriture de règles explicites
\end{itemize}
\newpage
