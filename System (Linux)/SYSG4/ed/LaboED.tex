\documentclass[french,10pt,A4]{report}
%\usepackage[latin1]{inputenc}
\usepackage[utf8]{inputenc}
\usepackage[francais]{babel}
% Modification des marges ------------------------------
\oddsidemargin -4mm % Decreases the left margin by 4mm
\textwidth 17cm %Sets text width across page = 17cm
\textheight 22cm %Sets text height up and down = 22cm
% -----------------------------------------------------
\usepackage[T1]{fontenc}
\usepackage{graphicx,color, caption2}
\usepackage{epsfig}
\usepackage{fancyhdr}
\pagestyle{fancy}
\usepackage{listings}
\usepackage{color}
\usepackage{makeidx}

% Valeurs par défaut le lstset
\lstset{language={},%C,Assembleur, TeX, tcl, basic, cobol, fortran, logo, make, pascal, perl, prolog, {}
	literate={â}{{\^a}}1 {ê}{{\^e}}1 {î}{{\^i}}1 {ô}{{\^o}}1 {û}{{\^u}}1
		 {ä}{{\"a}}1 {ë}{{\"e}}1 {ï}{{\"i}}1 {ö}{{\"o}}1 {ü}{{\"u}}1
		 {à}{{\`a}}1 {é}{{\'e}}1 {è}{{\`e}}1 {ù}{{\`u}}1 
		 {Â}{{\^A}}1 {Ê}{{\^E}}1 {Î}{{\^I}}1 {Ô}{{\^O}}1 {Û}{{\^U}}1
		 {Ä}{{\"A}}1 {Ë}{{\"E}}1 {Ï}{{\"I}}1 {Ö}{{\"O}}1 {Ü}{{\"U}}1
		 {À}{{\`A}}1 {É}{{\'E}}1 {È}{{\`E}}1 {Ù}{{\`U}}1,
	commentstyle=\scriptsize\ttfamily\slshape, % style des commentaires
	basicstyle=\scriptsize\ttfamily, % style par défaut
	keywordstyle=\scriptsize\rmfamily\bfseries,% style des mots-clés
	backgroundcolor=\color[rgb]{.95,.95,.95}, % couleur de fond : gris clair
	framerule=0.5pt,% Taille des bords
	frame=trbl,% Style du cadre
	frameround=tttt, % Bords arrondis 
	tabsize=3, % Taille des tabulations
%	extendedchars=\true, % Incompatible avec utf8 et literate
	inputencoding=utf8,
	showspaces=false, % Ne montre pas les espaces 
	showstringspaces=false, % Ne montre pas les espaces entre ''
	xrightmargin=-1cm, % Retrait gauche 
	xleftmargin=-1cm, % Retrait droit
	escapechar=°}  % Caractère d'échappement, permet des commandes latex dans la source
% -----------------------------------------------------
%\makeindex
\begin{document}
\lhead{Labo E.D. S.E 2$^eme$}
\rhead{Page \thepage}
\lfoot{\copyright J.C. Jaumain }
\rfoot{\today}
\cfoot{ }
\renewcommand{\footrulewidth}{0.4pt}

\setlength{\parindent}{0pt} % pas d'indentation

\lstset{frame=trBL}

\setcounter{tocdepth}{1}	% limiter les nivaux de table des matières
\setcounter{secnumdepth}{5}	% La numérotation des sections au maximum

\newcommand{\titre}{Titre du sujet}	% la variable contenant le titre du sujet

\thispagestyle{empty}

\title{\emph{Laboratoire\\\textbf{E.D. }}(ancien F.S)}
\author{Jaumain J-C}
%\date{15 janvier 2014}
\date{révision mba - Septembre 2017}
\maketitle
\tableofcontents
% ~\\[5cm]
% \flushright{Dis, papa, ça veut dire quoi "Formating drive c" ?}
% \flushleft{ }
\newpage
%
\chapter{Partitionnement et création de File System}
	\lstset{language=c}
\renewcommand{\titre}{\textcolor{blue}{ ED - LaboED 01-01 - Partitionnement DOS - fdisk}}

\lhead{ \titre }
\section{{\titre} }

\begin{tabular}{|l|l|}
\hline
Titre : 	& \titre \\\hline
Support : 	& OS 42.3 Leap\\\hline
Date :		& 08/2018 \\\hline
\end{tabular}

\subsection{Énoncé}

Être capable de partitionner un disque avec une table de type MBR. Être capable de décrire les partitions
La démonstration de cette manipulation ne sera pas automatisée. Vous ne devrez pas écrire de script Demo.

\subsection{Manipulation}

Cette partie du laboratoire se fait en tant qu'administrateur. Soyez très prudent et réfléchissez aux commandes que vous tapez !
N'utilisez le mode administrateur que quand c'est indispensable, c'est une mauvaise habitude de travailler en root à d'autres moments.\\
Demandez le mot de passe root au responsable du laboratoire.

\subsubsection{Partitionnons un disque}

fdisk permet de partitionner un disque avec une table de partitions MBR. Il faut lui donner comme argument le pilote associé au disque (/dev/sdb, /dev/sdc, ...) que l'on souhaite partitionner. Ici, nous utiliserons uniquement des stick usb de 32 GiB comme disque. Le partitionnement détruit toutes les données du support : il ne faut pas se tromper en donnant le nom du pilote (vous vous abstiendrez de partitionner /dev/sda donc) !!!. Pour travailler de façon sécurisée, vous procéderez comme suit :

\begin{enumerate}
\item N'utilisez que le stick usb que vous partitionnez, ôtez tous les autres
\item Placez le stick usb, attendez une seconde et tapez dmesg
\item Les dernières lignes affichées de cette commande vous donnent le nom du pilote associé au stick (sdb, sdc, sdd, ...), par exemple sdb.
\item Tapez fdisk /dev/sd? où ? est la lettre ci dessus
\item Affichez l'aide de la commande fdisk
\item Observez l'alignement de vos partitions
\item Quelle est la signification du champ Secteurs ?
\end{enumerate}

\subsubsection{Choix des partitions}

Il suffit maintenant de dire ce que vous souhaitez à fdisk. Par exemple, après avoir créé une nouvelle table de partitions DOS, créez sur ce device 3 partitions : 2 primaires, et une logique. Vous choisirez des tailles de partition <= 2GiB. réez une partition primaire de 1000 secteurs.
Affichez cette table dans fdisk et confirmez votre souhait en l'écrivant sur le stick.\\
Vérifiez l'état de votre table par la commande \emph{fdisk -l /dev/???}

\subsection{Lire la table des partitions via un programme c}

\subsubsection{Énoncé}

Écrire un programme qui, comme fdisk, affiche la tables des partitions d'un disque. Ce programme utilisera un seul argument : le nom du pilote de ce disque. On se contentera d'afficher les partitions primaires uniquement.
\subsubsection{Une solution}

\lstinputlisting{LaboED0101/SOURCES/LectMBR.c}

\subsection{Commentaires}

\begin{itemize}
\item La table des partitions se trouve dans le MBR du disque ou du stick usb. Il convient de ne pas l'écraser lors d'une copie d'un programme de chargement au risque de perdre tout le contenu du disque ou du stick
\item Un device est rattaché au système de fichiers via le répertoire /dev, il peut être lu comme si c'était un fichier. Les premiers 512 bytes lus seront le MBR du device.\\
\item Toutefois, comme la mise au point du programme est fastidieuse si on souhaite privilégier la sécurité : modification du programme en tant que user, test en tant que root. Il est plus simple de créer une fois un fichier représentant le MBR et de tester le programme sur base de ce fichier. Vous pouvez créer ce fichier via la commande administrateur \verb*dd : dd if=/dev/... of=MBR bs=512 count=1*\\
\item Dans le programme en c il faudra inhiber l'alignement sur les champs des structures. On utilise pour cela l'attribut packed.
\item Les partitions sont alignées sur une frontière de cylindre
\item Vous pouvez automatiser simplement fdisk en lui précisant les réponses dans un fichier et en utilisant la redirection de l'entrée standard < ou en utilisant la double redirection << dans le script Demo. 
\end{itemize}

\subsection{En roue libre}
\begin{itemize}
\item Adaptez le programme précédent pour afficher le type des partitions.
\item Que faudrat-il faire pour afficher les données de la partition logique ?
\end{itemize}
\newpage
 % 
	\lstset{language=c}
\renewcommand{\titre}{\textcolor{blue}{ ED - LaboED 01-02 - Partitionnement GPT - fdisk}}

\lhead{ \titre }
\section{{\titre} }

\begin{tabular}{|l|l|}
\hline
Titre : 	& \titre \\\hline
Support : 	& OS 42.3 Leap -installation Classique \\\hline
Date :		& 08/2018\\\hline
%Author:		& MBA\\\hline
\end{tabular}

\subsection{Énoncé}

Être capable de partitionner un disque avec une table de type GPT. 
Être capable de décrire les partitions. La démonstration de cette manipulation ne sera pas automatisée. Vous ne devrez pas écrire de script Demo. \\

\subsection{Manipulation}

Cette partie du laboratoire se fait en tant que administrateur. Soyez très prudent et réfléchissez aux commandes que vous tapez !
N'utilisez le mode administrateur que quand c'est indispensable, c'est une mauvaise habitude de travailler en root à d'autres moments.\\

\subsubsection{Partitionnons un disque}

fdisk permet de partitionner un disque avec une table de partitions GPT. Il faut lui donner comme argument le pilote associé au disque (/dev/sdb, /dev/sdc, ...) que l'on souhaite partitionner. Ici, nous utiliserons uniquement des stick usb de 32 Gib comme disque. Le partitionnement détruit toutes les données du support : il ne faut pas se tromper en donnant le nom du pilote (vous vous abstiendrez de partitionner /dev/sda donc) !!!. Pour travailler de façon sécurisée, vous procéderez comme suit :

\begin{enumerate}
\item N'utilisez que le stick usb que vous partitionnez, ôtez tous les autres
\item Placez le stick usb, attendez une seconde et tapez dmesg suivi de lsblk.
\item Les dernières lignes affichées par la commande dmesg vous donnent le nom du pilote associé au stick (sdb, sdc, sdd, ...), par exemple sdb.
\item Tapez \emph{fdisk /dev/sd?} où ? est la lettre ci dessus
\item Affichez l'aide de la commande fdisk
\end{enumerate}

\subsubsection{Choix des partitions}

Il suffit maintenant de dire ce que vous souhaitez à fdisk. Par exemple, après avoir créé une nouvelle table de partitions GPT, obtenez le partitionnement suivant :

\begin{itemize}
\item partition 1 de 1Gib
\item partition 2 de 2Gib
\end{itemize}

Affichez cette table avec fdisk et confirmez votre souhait en l'écrivant sur le stick.

\subsection{Lire la table des partitions via un programme c}

\subsubsection{Énoncé}

Écrire un programme qui, comme fdisk, affiche la tables des partitions GPT. Ce programme lira également l'entête secondaire et le descripteur secondaire de la partition1. Ce programme utilisera un seul argument : le nom du pilote de ce disque.
\subsubsection{Une solution}

\lstinputlisting{LaboED0102/SOURCES/LectGPT.c}

\subsection{Commentaires}
\begin{itemize}
\item La table des partitions se trouve dans les secteurs suivants du disque ou du stick usb.
\item L'attribut packed sert à forcer l'alignement des champs d'une structure.
\item L'entète de partidion secondaire est le dernier secteur du disque.
\end{itemize}
\subsection{En roue libre}
Dans le code c fourni, réalisez la restauration du descripteur de la partition 2 à partir du descripteurs secondaire qui se trouve en fin de disque. Soyez très prudent dans la spécification du nom de device pour cette opérationi d'écriture, une erreur peut endommager votre installation. Affichez également le type de la partition primaire décrite dans le MBR.
\newpage
 %
	\lstset{language=c}
\renewcommand{\titre}{\textcolor{blue}{ ED - LaboED 01-03 - Structure F.S. - mkfs}}

\lhead{ \titre }
\section{{\titre} }

\begin{tabular}{|l|l|}
\hline
Titre : 	& \titre \\\hline
Support : 	& OS 43.2 Leap - Installation Classique \\\hline
Date :		& 08/2018 \\\hline
\end{tabular}

\subsection{Énoncé}

Être capable de créer un F.S. et de l'utiliser.\\
Cet exercice de laboratoire ne sera pas automatisé. Vous ne devez pas créer de script Demo.

\subsection{Manipulation}

Cette partie du laboratoire se fait en tant que administrarteur. Soyez très prudent et réfléchissez aux commandes que vous tapez !
N'utilisez le mode administrateur que quand c'est indispensable.\\
Demandez le mot de passe root au responsable du laboratoire.

\subsubsection{Partitionner un disque et formater les partitions}

Vous savez partitionner un disque. Pour que ces partitions soient utilisables, il faut y inscrire la structure d'un F.S. de votre choix. (=formater les partitions)

Adaptez sdb, sdc, sdd, ... suivant votre situation.

A l'aide de fdisk, obtenez deux partitions primaires et une logique, toutes de taille <= 2Gib\\
vérifiez le partitionnement et formatez quelques partitions en adaptant la lettre ? :

\begin{lstlisting}
fdisk -l
mkfs.vfat /dev/sd?1  # fat 16
mkfs.ext2 /dev/sd?2
mkfs.ext2 /dev/sd?5
fdisk -l
\end{lstlisting}

La table des partitions tient-elle compte de vos changements de type ?

\subsection{Commentaires}
\begin{itemize}
\item fdisk permet de partitionner, mkfs permet de formater.
\item ces commandes nécessitent les droits administrateur
\end{itemize}

\newpage
 %
	\lstset{language=c}
\renewcommand{\titre}{\textcolor{blue}{ ED - LaboED 01-04 - fdisk automatisé}}

\lhead{ \titre }
\section{{\titre} }

\begin{tabular}{|l|l|}
\hline
Titre : 	& \titre \\\hline
Support : 	& OS 43.2 Leap - Installation Classique \\\hline
Date :		& 08/2018\\\hline
\end{tabular}

\subsection{Énoncé}

Automatiser le partitionnement d'un disque. \\


N'utilisez que le stick usb que vous partitionnez, ôtez tous les autres

\subsection{Commentaires}

\begin{itemize}
\item Vous pouvez automatiser simplement fdisk dans un script en lui précisant les réponses avec la double direction d'entrée <<. En ligne de commande cela peut également être fait par la redirection de l'entrée standard < . En effet c'est sur stdin que fdisk lit vos commandes. 
\item Soyez toujours très prudents quand vous effectuez des manipulations en administrateur. Le script Demo devra être exécuté en administrateur. Il faudra vérifier cela, de plus le nom du pilote/device n'est pas identique sur chacune des machines. Une erreur de nom peut compromettre votre système d'exploitation !
\item Demandez le mot de passe root au responsable du laboratoire.
\end{itemize}

\subsection{En roue libre}
Adaptez le script Demo pour automatiser le partitionnement suivant

\begin{itemize}
\item création d'un nouvelle table de partitions DOS contenant :
\item une partition primaire 1 < 1Gib
\item une partition primaire 2 étendue de tout le reste du stick usb
\item une partition logique 5 < 1 Gib
\end{itemize}
Les valeurs par défaut obtenues par la touche <CR>  correspondent à une ligne blanche dans le fichier de commandes 
\newpage
 %
\chapter{Ext - Inodes et blocs}
	\lstset{language=c}
\renewcommand{\titre}{\textcolor{blue}{ ED - LaboED 02-01 - Inodes - stat,lstat}}

\lhead{ \titre }
\section{{\titre} }

\begin{tabular}{|l|l|}
\hline
Titre : 	& \titre \\\hline
Support : 	& OS 42.3 Leap - Installation Classique \\\hline
Date :		& 08/2018 \\\hline
\end{tabular}

\subsection{Énoncé}

Écrivez un programme qui affiche le contenu de l'inode d'un fichier passé en argument.

\subsection{Une solution}

\lstinputlisting{LaboED0201/SOURCES/ContenuInode.c}

\subsection{Commentaires}

\begin{itemize}
\item man 2 stat donne toutes les informations nécessaires pour l'écriture de ce programme.
\item Le lien software lienS est déréférencé par l'appel système fstat.
\item lstat donne des informations sur le fichier lien.
\item Le numéro du propriétaire est mémorisé dans l'inode du fichier. L'association numéro - nom du propriétaire se fait à l'aide du fichier /etc/passwd. Si on copie un fichier d'un PC vers un autre, il peut changer de propriétaire !!!
\item Le nombre de blocs est souvent exprimé en secteurs de 512 bytes. du -h (alias pour du) donne cette information en K (Kib=1024 bytes)
\end{itemize}

\subsection{En roue libre}

Adaptez l'exercice :\\
\begin{itemize}
\item en y ajoutant un lien hardware
\item en y ajoutant le détail pour le lien software.
\end{itemize}
Créez ensuite le fichier archive tar pour cet exercice uniquement en adaptant la structure et les fichiers latex fournis.\\
Montrez le fichier tar à votre professeur.
\newpage
 % contenu d'un inode
	\lstset{language=c}
\renewcommand{\titre}{\textcolor{blue}{ ED - LaboED 02-02 - Fichiers creux - lseek, ls, du -h}}

\lhead{ \titre }
\section{{\titre} }

\begin{tabular}{|l|l|}
\hline
Titre : 	& \titre \\\hline
Support : 	& OS 42.3 Leap - Installation Classique \\\hline
Date :		& 08/2018 \\\hline
\end{tabular}

\subsection{Énoncé}

Écrivez un programme qui crée un fichier contenant 'Hello world' au début du fichier et 'Bye world' en position 100000 grâce à l'appel système lseek.

\subsection{Une solution}

\lstinputlisting{LaboED0202/SOURCES/FichierCreux.c}

\subsection{Commentaires}

\begin{itemize}
\item Ce fichier a une taille de 100009 bytes et occupe 3 blocs de 1K sur le disque. Le premier bloc référencé par le $1^{er}$ pointeur contient Hello world, le deuxième bloc, référencé par le $13^{eme}$ pointeur, contient 256 pointeurs. Un de ceux-ci référence le bloc qui contient 'Bye world'. Les pointeurs 2-12,14 et 15 contiennent 0 car il ne pointent vers aucun bloc.
\item Sur un F.S. de type FAT, le fichier occupera réellement 100K.
\item Pour connaître la taille des blocs de votre FS, il suffit de créer un fichier de un byte et de regarder son occupation sur le disque. Elle est forcément de un bloc.
\item Ce programme peut être utilisé pour créer des fichiers de grande taille (jusqu'à 2 Gib sur un système 32 bits) qui occupe très peu de place sur le disque. Mais ne copiez pas un tel fichier sur un stick usb en FAT !!!
\item Ce programme, exécuté sur un F.S. de type ext2, ne prendra que quelques secondes car il écrit peu sur le disque. Inversement, ce même programme, exécuté sur un F.S. de type FAT, prendra plusieurs minutes car il écrit beaucoup sur le disque.
\end{itemize}

\subsection{En roue libre}
Vérifiez et justifiez le comportement si on ajoute le flag O\_APPEND dans le paramètre flags du open. L'appel système lseek retourne-t-il une erreur dans ce cas ?

\newpage
 % fichier creux
	\lstset{language=c}
\renewcommand{\titre}{\textcolor{blue}{ ED - LaboED 02-03 - Structure F.S. - mkfs, debugfs}}

\lhead{ \titre }
\section{{\titre} }

\begin{tabular}{|l|l|}
\hline
Titre : 	& \titre \\\hline
Support : 	& OS 42.3 Leap - Installation Classique \\\hline
Date :		& 08/2018 \\\hline
\end{tabular}

\subsection{Énoncé}

Être capable de créer un F.S. et de l'utiliser.\\
Cet exercice de laboratoire ne sera pas automatisé. Vous ne devez pas créer de script Demo.

\subsection{Manipulation}

Cette partie du laboratoire se fait en tant que administrateur. Soyez très prudent et réfléchissez aux commandes que vous tapez !
N'utilisez le mode administrateur que quand c'est indispensable.\\
Demandez le mot de passe root au responsable du laboratoire.

\subsubsection{Partitionner un disque et formater les partitions}

Vous savez partitionner un disque. Pour que ces partitions soient utilisables, il faut y inscrire la structure d'un F.S. de votre choix. (=formater les partitions)

Adaptez sdb, sdc, sdd, ... suivant votre situation.

A l'aide de fdisk, obtenez deux partition de même taille <= 2Gib\\
formatez-la première en ext2.

\begin{lstlisting}
fdisk -l
mkfs.ext2 /dev/sd??
\end{lstlisting}

En lisant l'output fourni avec la commande mkfs.ext2, répondez aux questions suivantes : 
\begin{enumerate}
\item Combien de fichiers pourriez-vous avoir au maximum sur ce F.S. ?
\item Quelle est la taille oar défaut d'un bloc ?
\item Á combien de blocs avez-vous droit ?
\item Formatez en ext2 l'autre partition en fixant la taille de bloc à 1Kib (man mkfs.ext2) et comparez les valeurs obtenues
\end{enumerate}

Copier dans ce deuxième F.S. un petit(<1K),  un gros(<10K) et un très gros(>7M) fichiers. Pour ce dernier, les commandes \verb+yes+ (attention, ça va très vite ...) ls -lR ... redirigées vous permet de créer cela facilement.
\begin{lstlisting}
mkdir disk		# créer un point de montage
mount /dev/sdb2 disk	# monter la partition choisie
cp petit disk/
cp gros disk/
cp tresgros disk/
\end{lstlisting}

Utilisez debugfs pour décrire le F.S.

\begin{lstlisting}
/sbin/debugfs /dev/sdb2
help # donne la liste des commandes
quit # sortir de debugfs
\end{lstlisting}

\begin{enumerate}
\item (debugfs)ls -l : quels sont les numéros d'inode de petit et de gros ?
\item (debugfs)stat petit : Combien de blocs petit utilise-t'il ? et gros et tresgros ?
\item Quels sont les numéros de blocs utilisés par petit et gros et tresgros ?
\item (debugfs)icheck : vérifier qu'un bloc appartient bien à l'inode supposé
\item (debugfs)ncheck : vérifier qu'un inode représente bien le fichier supposé
\item (debugfs)ffb : le premier bloc libre
\item (debugfs)ffi : le premier inode libre
\item (debugfs)undel : restaure un fichier
\item (debugfs)stats : Le contenu du superbloc, combien d'inodes y a-t-il par groupe ?
\item (debugfs)quit : pour terminer
\item Pourriez-vous dessiner ce F.S. avec les bons numéros d'inodes et de blocs ?
\end{enumerate}


\subsection{Commentaires}

\begin{itemize}
\item debugfs permet de lire et écrire la structure d'un F.S. de type ext.
\item debugfs permet de voir le chaînage des blocs dans le détail
\end{itemize}

\newpage
 % debugfs
\chapter{Ext - Répertoires}
	\lstset{language=c}
\renewcommand{\titre}{\textcolor{blue}{ ED - LaboED 03-01 - Répertoires - opendir, readdir, lstat}}

\lhead{ \titre }
\section{{\titre} }

\begin{tabular}{|l|l|}
\hline
Titre : 	& \titre \\\hline
Support : 	& OS 42.3 Leap - Installation Classique \\\hline
Date :		& 08/2018 \\\hline
\end{tabular}

\subsection{Énoncé}

Écrivez un programme qui affiche les noms et les numéros d'inode des objets de votre répertoire.

\subsection{Une solution}

\lstinputlisting{LaboED0301/SOURCES/ContenuRepertoire.c}

\subsection{Commentaires}

\begin{itemize}
\item En unix, un fichier caché est un fichier dont le nom commence par un .
\item Pour obtenir la liste des fichiers réguliers, il faut ouvrir chaque fichier et vérifier son mode (soit via stat, soit via open et fstat : voir man 2 stat)
\item Pour obtenir une liste récursive d'un répertoire, il faut écrire un programme récursif sans oublier d'exclure les répertoires . et .. (sinon boucle infinie !)
\item Il est très difficile d'estimer la place libre d'un minidisque. L'espace total - la somme des occupations des différents fichiers (du) donne un nombre. Il reste possible d'écrire un fichier qui aura une taille supérieure à ce nombre (fichier creux). df donne un résumé des minidisques montés et de la place qu'il y reste. 
\end{itemize}

\subsection{En roue libre}
Dans un dossier contenant des sous-dossiers non vides, écrivez le code en c qui réalise la même chose que la commande ls *. 
\newpage
 % parcours de répertoire
	\lstset{language=c}
\renewcommand{\titre}{\textcolor{blue}{ ED - LaboED 03-02 - Droits des fichiers  - SUID}}

\lhead{ \titre }
\section{{\titre} }

\begin{tabular}{|l|l|}
\hline
Titre : 	& \titre \\\hline
Support : 	& OS 42.3 Leap - Installation Classique \\\hline
Date :		& 08/2018 \\\hline
\end{tabular}

\subsection{Énoncé}

Comme administrateur, créez un fichier appelé Confidentiel qui contient le texte 'LE SECRET'. 
Écrivez un programme Conf qui affiche "Contenu du fichier Confidentiel : ", suivi du contenu du fichier appelé Confidentiel. 
Modifiez les droits de telle façon que personne d'autre que l'utilisateur root ne puisse afficher le contenu du fichier appelé Confidentiel mais que tout le monde puisse exécuter le programme Conf et afficher ainsi le contenu du fichier Confidentiel.

\subsection{Une solution}

\lstinputlisting{LaboED0302/SOURCES/Conf.c}

\subsection{Commentaires}

\begin{itemize}
\item La commande \verb+chmod 4755 Conf + positionne le bit SUID du programme Conf.  Le \emph{effective uid (euid)} de l'utilisateur qui exécute Conf sera modifié en le uid du propriétaire de Conf (root). Tout utilisateur aura le droit de voir le contenu de votre fichier  Confidentiel uniquement en exécutant le programme Conf. Ce droit lui sera refusé si il veut visionner le fichier par d'autres moyens (par exemple par \emph{cat Confidentiel}).
\item Quels droits a l'exécutable \verb+/bin/cat+ ?
\item Quelle conséquence aurait la commande \verb+chmod 4755 /bin/cat+, exécutée par un administrateur ?
\item ATTENTION !!! Veillez à remettre les bons droits à \verb+cat+ si vous les avez modifiés !
\item Quittez le login administrateur.
\end{itemize}
\newpage
 % droits + SUID, SGID
\chapter{Commandes filtre}
	\lstset{language=c}
\renewcommand{\titre}{\textcolor{blue}{ ED - LaboED 04-01 - Commande filtre, cat sans argument en c }}

\lhead{ \titre }
\section{{\titre} }

\begin{tabular}{|l|l|}
\hline
Titre : 	& \titre \\\hline
Support : 	& OS 42.3 Leap - Installation Classique \\\hline
Date :		& 08/2018 \\\hline
\end{tabular}

\subsection{Énoncé}
Une commande filtre en Unix est une commande qui lit des données sur stdin et les restitue transformées sur stdout.\emph{cat}, utilisée sans arguments est la plus simple des "commandes filtre", elle n'effectue aucune transformation de données. Une commande filtre peut être utilisée à gauche et à droite d'un pipe.\\ 
Testez la commande cat sans arguments.\\
Ce programme très simple ne fait qu'écrire sur la sortie standard (stdout) ce qu'il lit sur l'entrée standard (stdin) jusqu'à la fin de celle-ci. Dans le cas du clavier, la fin de stdin est provoquée par la combinaison de touches CTRL-D (très différent de CTRL-C) qui tue les programmes liés au terminal.\\
\emph{cat} et Mcat sans argument permettent d'afficher le contenu d'un fichier (<) ou de créer un fichier avec un contenu introduit au clavier (>). \\
Réécrivez la commande cat sans arguments en c et baptisez-la Mcat. Votre Mcat doit pouvoir réaliser les fonctions équivalentes de \emph{cat} dans les situations suivantes : Mcat, ls | Mcat, Mcat > fichier, ls | Mcat >> fichier , Mcat < fichier

\subsection{Une solution}

\lstinputlisting{LaboED0401/SOURCES/Mcat.c}

\subsection{Commentaires}

\begin{itemize}
\item 
Nous avons découvert que les redirections et les pipes ne sont pas traités comme des arguments, ils sont donc gérés au préalable par le shell. Dans l'énoncé, on fait donc toujours appel à Mcat sans argument. 
\item La logique du programme Mcat est de recopier sur stdout tout ce qu'on lit sur stdin jusqu'à la fin du fichier stdin. 
\item La fin du fichier stdin "clavier" se fait par CTRL-D.
\item La lecture et l'écriture se font par appel au S.E. Ceci est beaucoup plus visible en assembleur.
\item C'est le S.E. qui gère les buffers d'E/S. Le programme ne sera pas plus rapide en utilisant un buffer plus grand.
\item Une commande qui utilise stdin et stdout est appelée un filtre. cat est un filtre qui 'laisse tout passer'.
\item Chaque commande que vous utilisez peut être parfois vue comme un filtre. Cela dépend de la commande et de la façon de l'utiliser. Seuls les filtres peuvent être placés entre deux pipes.
\end{itemize}

\subsection{En roue libre}
Écrivez le script Demo qui montre que le code fait bien ce qu'on attend de lui. Référez-vous à l'énoncé.
\newpage
 % cat sans arg en c
	\lstset{language=c}
\renewcommand{\titre}{\textcolor{blue}{ ED - LaboED 04-02 - cat avec arguments en c }}

\lhead{ \titre }
\section{{\titre} }

\begin{tabular}{|l|l|}
\hline
Titre : 	& \titre \\\hline
Support : 	& OS 42.3 Leap - Installation Classique \\\hline
Date :		& 08/2018 \\\hline
\end{tabular}

\subsection{Énoncé}

Réécrivez la commande cat en c et baptisez la McatArgs. Votre Mcat doit pouvoir réaliser les fonctions équivalentes de cat sans option.

\subsection{Une solution}

\lstinputlisting{LaboED0402/SOURCES/McatArgs.c}

\subsection{Commentaires}

\begin{itemize}
\item Les commandes telles que cat ne font pas partie du S.E. Elles sont fournies pour éviter que chacun reprogramme de telles commandes. Elles ne sont pas indispensables au fonctionnement du S.E.
\item L'écriture d'une commande filtre suit quasi toujours le même canevas. Un main qui appelle un traitement pour chaque argument passé. Si il n'y a pas d'argument, le seul fichier traité est stdin.
\item Cette commande fonctionne aussi avec des wildcards. C'est le shell qui interprète les wildcards et transforme le résultat en arguments. Pour cat, il n'y a pas de différence entre cat * et cat f1 f2 f3 si ces trois fichiers sont les seuls du répertoire courant.
\item Vous pouvez obtenir une liste des commandes qui existent en regardant le contenu des répertoires /bin, /usr/bin, /sbin, /usr/sbin, ... et leur rôle à l'aide des pages de manuel, chapitre 1.
\end{itemize}

\subsection{En roue libre}
Adaptez cet exercice pour réécrire la commande filtre head (man head)

\newpage
 % cat avec arg en c
\chapter{Exemples d'exercices}
	\lhead{ Labo ED - Exercices}

\begin{list}{+}{}

\item ED005: 
Réécrivez la commande head en c et baptisez-le Mhead. Votre Mhead doit pouvoir réaliser les fonctions équivalentes de head dans les situations suivantes : Mhead, ls | Mhead, Mhead > fichier, ps | Mhead >>fichier. Utilisez man pour comprendre ce que fait la commande head.

\item ED009: 
Réécrivez la commande cat en c et baptisez-le Mcat.
Donnez ensuite toutes les étapes que vous devez effectuer pour créer un fichier confidentiel contenant 'phrase confidentielle' que VOUS SEUL pouvez lire / écrire.
Donner les droits au programme Mcat de telle façon que tous puissent exécuter Mcat qui permettra de visualiser TOUS les fichiers que VOUS pouvez lire notamment le fichier confidentiel.

\item ED010: 
Imaginez un exemple qui prouve que les droits d'accès sont vérifiés uniquement à l'open d'un fichier.

\item ED011: 
Écrivez un programme qui crée un nouveau fichier. Ce fichier contient seulement un 'a' en position 0, un 'b' en position 1000 et un 'c' en position 10000. Imaginez une technique pour connaître la taille des blocs de votre système de fichiers et déduisez la place occupée par le fichier que vous venez de créer. Vérifiez avec la commande du. Même question si vous écrivez 'a' en 0, 'b' en 70000 ? Même question si vous écrivez 'a' en 0, 'b' en 500000 ? Même question si vous écrivez 'a' en 0, 'b' en 1000000000 ? Expliquez en détail la réponse obtenue.

\item ED013: 
Écrivez un programme qui donne toute l'arborescence qui se trouve en dessous d'un répertoire donné en paramètre. Vous le baptiserez Mtree. Vous afficherez uniquement les noms des répertoires, pas les noms des fichiers. Un décalage d'un espace vers la droite symbolisera qu'un répertoire est compris dans un autre répertoire. Par exemple :
\begin{verbatim}
\
 home
  g12345
   projet
   systeme
  g23456
   projet
...
\end{verbatim}


\item ED020: 
Un processus exécute le code correspondant à ces 3 appels :
\begin{lstlisting}[frame=trBL]{}
fd1 = open(Nomf, OFlags);
fd2 = dup (fd1);
fd3 = open(Nomf,OFlags);
\end{lstlisting}
Dessinez la table des descripteurs de fichier correspondant à ce processus après l'exécution des lignes ci-dessus. Écrivez un programme qui montre que votre réponse est correcte.


\item ED023: 
LienFich est un lien vers le fichier Fich. Comment faites-vous pour savoir s'il s'agit d'un lien soft ou d'un lien hard ?


\item ED027: 
Décrivez le comportement de ce programme:
\begin{lstlisting}[frame=trBL]{}
main( )
{	int f;
	f=open("fichier",O_WRONLY|O_CREAT,0777);
	dup2(f,1);
	close(f);
	printf("Hello world\n");
	exit(0);
}
\end{lstlisting}


\item ED037: 
Voici une série de commandes effectuées par l'utilisateur root
\lstset{language=bash,frame=trBL}
\begin{lstlisting}[frame=trBL]{}
mkfs.ext /dev/fd0
mount /dev/fd0 flop
./CreeFich
cd flop
ln -s fich fs
ln fich fh
\end{lstlisting}

Où le programme CreeFich est le résultat de la compilation du programme CreeFich.c :

\lstset{language=c,frame=trBL}
\begin{lstlisting}[frame=trBL]{}
int main() {
	char * buffer = "Hello";
	int h;
	h=open("flop/fich",O_WRONLY|O_CREAT,0644);
	write(h,buffer,5);
	lseek(h,12000,SEEK_SET);
	write(h,buffer,5);
	close(h);
	exit(0);
}
\end{lstlisting}

Réalisez une description complète et précise du F.S. ext qui résulte de l'exécution de ces commandes sachant que la taille des blocs est de 1K.


\item ED048: 
Créer un minidisk contenant un FS de type ext2. Copiez sur ce minidisk un répertoire qui contient deux fichiers simples. Décrivez le minidisk correspondant (tableau d'inodes tableau de blocs) en utilisant les résultats donnés par debugfs.  


\item ED057:
Écrivez un programme qui affiche les caractéristiques des partitions primaires d'un disque en précisant si ces partitions sont bootables.  


\item ED064: 
Écrivez un programme qui construit une table contenant deux champs. Le premier champ est la taille d'un fichier contenant uniquement des caractères 'y'. Le deuxième champ est le nombre de blocs utilisés pour mémoriser ce fichier sur le disque. Exécutez ce programme avec 10, 100, 200, 300, 1000, 2000, 3000, 10000, 20000, 30000, 100000, 200000, 300000,... caractères. Expliquez le résultat obtenu. 
 

\item ED073:
Réécrivez la commande cp f1 f2 en c et baptisez-le Mcp. Votre Mcp ne doit réaliser que la copie d'un seul fichier vers un seul fichier. Les noms sont donnés en paramètre.

\item ED079:
Réécrivez la commande rm en c et baptisez-le Mrm. Votre Mrm doit fonctionner avec un ou plusieurs fichiers donnés en paramètre. 

\item ED086:
Écrivez, en c, l'équivalent de la commande find /usr -perm 4755.

\end{list}



\end{document}
