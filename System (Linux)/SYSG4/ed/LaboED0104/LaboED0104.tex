\lstset{language=c}
\renewcommand{\titre}{\textcolor{blue}{ ED - LaboED 01-04 - fdisk automatisé}}

\lhead{ \titre }
\section{{\titre} }

\begin{tabular}{|l|l|}
\hline
Titre : 	& \titre \\\hline
Support : 	& OS 43.2 Leap - Installation Classique \\\hline
Date :		& 08/2018\\\hline
\end{tabular}

\subsection{Énoncé}

Automatiser le partitionnement d'un disque. \\


N'utilisez que le stick usb que vous partitionnez, ôtez tous les autres

\subsection{Commentaires}

\begin{itemize}
\item Vous pouvez automatiser simplement fdisk dans un script en lui précisant les réponses avec la double direction d'entrée <<. En ligne de commande cela peut également être fait par la redirection de l'entrée standard < . En effet c'est sur stdin que fdisk lit vos commandes. 
\item Soyez toujours très prudents quand vous effectuez des manipulations en administrateur. Le script Demo devra être exécuté en administrateur. Il faudra vérifier cela, de plus le nom du pilote/device n'est pas identique sur chacune des machines. Une erreur de nom peut compromettre votre système d'exploitation !
\item Demandez le mot de passe root au responsable du laboratoire.
\end{itemize}

\subsection{En roue libre}
Adaptez le script Demo pour automatiser le partitionnement suivant

\begin{itemize}
\item création d'un nouvelle table de partitions DOS contenant :
\item une partition primaire 1 < 1Gib
\item une partition primaire 2 étendue de tout le reste du stick usb
\item une partition logique 5 < 1 Gib
\end{itemize}
Les valeurs par défaut obtenues par la touche <CR>  correspondent à une ligne blanche dans le fichier de commandes 
\newpage
