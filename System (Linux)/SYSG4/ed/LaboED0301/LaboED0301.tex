\lstset{language=c}
\renewcommand{\titre}{\textcolor{blue}{ ED - LaboED 03-01 - Répertoires - opendir, readdir, lstat}}

\lhead{ \titre }
\section{{\titre} }

\begin{tabular}{|l|l|}
\hline
Titre : 	& \titre \\\hline
Support : 	& OS 42.3 Leap - Installation Classique \\\hline
Date :		& 08/2018 \\\hline
\end{tabular}

\subsection{Énoncé}

Écrivez un programme qui affiche les noms et les numéros d'inode des objets de votre répertoire.

\subsection{Une solution}

\lstinputlisting{LaboED0301/SOURCES/ContenuRepertoire.c}

\subsection{Commentaires}

\begin{itemize}
\item En unix, un fichier caché est un fichier dont le nom commence par un .
\item Pour obtenir la liste des fichiers réguliers, il faut ouvrir chaque fichier et vérifier son mode (soit via stat, soit via open et fstat : voir man 2 stat)
\item Pour obtenir une liste récursive d'un répertoire, il faut écrire un programme récursif sans oublier d'exclure les répertoires . et .. (sinon boucle infinie !)
\item Il est très difficile d'estimer la place libre d'un minidisque. L'espace total - la somme des occupations des différents fichiers (du) donne un nombre. Il reste possible d'écrire un fichier qui aura une taille supérieure à ce nombre (fichier creux). df donne un résumé des minidisques montés et de la place qu'il y reste. 
\end{itemize}

\subsection{En roue libre}
Dans un dossier contenant des sous-dossiers non vides, écrivez le code en c qui réalise la même chose que la commande ls *. 
\newpage
