\lstset{language=c}
\renewcommand{\titre}{\textcolor{blue}{ ED - LaboED 02-03 - Structure F.S. - mkfs, debugfs}}

\lhead{ \titre }
\section{{\titre} }

\begin{tabular}{|l|l|}
\hline
Titre : 	& \titre \\\hline
Support : 	& OS 42.3 Leap - Installation Classique \\\hline
Date :		& 08/2018 \\\hline
\end{tabular}

\subsection{Énoncé}

Être capable de créer un F.S. et de l'utiliser.\\
Cet exercice de laboratoire ne sera pas automatisé. Vous ne devez pas créer de script Demo.

\subsection{Manipulation}

Cette partie du laboratoire se fait en tant que administrateur. Soyez très prudent et réfléchissez aux commandes que vous tapez !
N'utilisez le mode administrateur que quand c'est indispensable.\\
Demandez le mot de passe root au responsable du laboratoire.

\subsubsection{Partitionner un disque et formater les partitions}

Vous savez partitionner un disque. Pour que ces partitions soient utilisables, il faut y inscrire la structure d'un F.S. de votre choix. (=formater les partitions)

Adaptez sdb, sdc, sdd, ... suivant votre situation.

A l'aide de fdisk, obtenez deux partition de même taille <= 2Gib\\
formatez-la première en ext2.

\begin{lstlisting}
fdisk -l
mkfs.ext2 /dev/sd??
\end{lstlisting}

En lisant l'output fourni avec la commande mkfs.ext2, répondez aux questions suivantes : 
\begin{enumerate}
\item Combien de fichiers pourriez-vous avoir au maximum sur ce F.S. ?
\item Quelle est la taille oar défaut d'un bloc ?
\item Á combien de blocs avez-vous droit ?
\item Formatez en ext2 l'autre partition en fixant la taille de bloc à 1Kib (man mkfs.ext2) et comparez les valeurs obtenues
\end{enumerate}

Copier dans ce deuxième F.S. un petit(<1K),  un gros(<10K) et un très gros(>7M) fichiers. Pour ce dernier, les commandes \verb+yes+ (attention, ça va très vite ...) ls -lR ... redirigées vous permet de créer cela facilement.
\begin{lstlisting}
mkdir disk		# créer un point de montage
mount /dev/sdb2 disk	# monter la partition choisie
cp petit disk/
cp gros disk/
cp tresgros disk/
\end{lstlisting}

Utilisez debugfs pour décrire le F.S.

\begin{lstlisting}
/sbin/debugfs /dev/sdb2
help # donne la liste des commandes
quit # sortir de debugfs
\end{lstlisting}

\begin{enumerate}
\item (debugfs)ls -l : quels sont les numéros d'inode de petit et de gros ?
\item (debugfs)stat petit : Combien de blocs petit utilise-t'il ? et gros et tresgros ?
\item Quels sont les numéros de blocs utilisés par petit et gros et tresgros ?
\item (debugfs)icheck : vérifier qu'un bloc appartient bien à l'inode supposé
\item (debugfs)ncheck : vérifier qu'un inode représente bien le fichier supposé
\item (debugfs)ffb : le premier bloc libre
\item (debugfs)ffi : le premier inode libre
\item (debugfs)undel : restaure un fichier
\item (debugfs)stats : Le contenu du superbloc, combien d'inodes y a-t-il par groupe ?
\item (debugfs)quit : pour terminer
\item Pourriez-vous dessiner ce F.S. avec les bons numéros d'inodes et de blocs ?
\end{enumerate}


\subsection{Commentaires}

\begin{itemize}
\item debugfs permet de lire et écrire la structure d'un F.S. de type ext.
\item debugfs permet de voir le chaînage des blocs dans le détail
\end{itemize}

\newpage
