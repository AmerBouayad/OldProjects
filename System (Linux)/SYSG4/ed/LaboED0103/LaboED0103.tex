\lstset{language=c}
\renewcommand{\titre}{\textcolor{blue}{ ED - LaboED 01-03 - Structure F.S. - mkfs}}

\lhead{ \titre }
\section{{\titre} }

\begin{tabular}{|l|l|}
\hline
Titre : 	& \titre \\\hline
Support : 	& OS 43.2 Leap - Installation Classique \\\hline
Date :		& 08/2018 \\\hline
\end{tabular}

\subsection{Énoncé}

Être capable de créer un F.S. et de l'utiliser.\\
Cet exercice de laboratoire ne sera pas automatisé. Vous ne devez pas créer de script Demo.

\subsection{Manipulation}

Cette partie du laboratoire se fait en tant que administrarteur. Soyez très prudent et réfléchissez aux commandes que vous tapez !
N'utilisez le mode administrateur que quand c'est indispensable.\\
Demandez le mot de passe root au responsable du laboratoire.

\subsubsection{Partitionner un disque et formater les partitions}

Vous savez partitionner un disque. Pour que ces partitions soient utilisables, il faut y inscrire la structure d'un F.S. de votre choix. (=formater les partitions)

Adaptez sdb, sdc, sdd, ... suivant votre situation.

A l'aide de fdisk, obtenez deux partitions primaires et une logique, toutes de taille <= 2Gib\\
vérifiez le partitionnement et formatez quelques partitions en adaptant la lettre ? :

\begin{lstlisting}
fdisk -l
mkfs.vfat /dev/sd?1  # fat 16
mkfs.ext2 /dev/sd?2
mkfs.ext2 /dev/sd?5
fdisk -l
\end{lstlisting}

La table des partitions tient-elle compte de vos changements de type ?

\subsection{Commentaires}
\begin{itemize}
\item fdisk permet de partitionner, mkfs permet de formater.
\item ces commandes nécessitent les droits administrateur
\end{itemize}

\newpage
