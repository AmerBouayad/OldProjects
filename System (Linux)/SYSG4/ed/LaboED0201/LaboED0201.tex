\lstset{language=c}
\renewcommand{\titre}{\textcolor{blue}{ ED - LaboED 02-01 - Inodes - stat,lstat}}

\lhead{ \titre }
\section{{\titre} }

\begin{tabular}{|l|l|}
\hline
Titre : 	& \titre \\\hline
Support : 	& OS 42.3 Leap - Installation Classique \\\hline
Date :		& 08/2018 \\\hline
\end{tabular}

\subsection{Énoncé}

Écrivez un programme qui affiche le contenu de l'inode d'un fichier passé en argument.

\subsection{Une solution}

\lstinputlisting{LaboED0201/SOURCES/ContenuInode.c}

\subsection{Commentaires}

\begin{itemize}
\item man 2 stat donne toutes les informations nécessaires pour l'écriture de ce programme.
\item Le lien software lienS est déréférencé par l'appel système fstat.
\item lstat donne des informations sur le fichier lien.
\item Le numéro du propriétaire est mémorisé dans l'inode du fichier. L'association numéro - nom du propriétaire se fait à l'aide du fichier /etc/passwd. Si on copie un fichier d'un PC vers un autre, il peut changer de propriétaire !!!
\item Le nombre de blocs est souvent exprimé en secteurs de 512 bytes. du -h (alias pour du) donne cette information en K (Kib=1024 bytes)
\end{itemize}

\subsection{En roue libre}

Adaptez l'exercice :\\
\begin{itemize}
\item en y ajoutant un lien hardware
\item en y ajoutant le détail pour le lien software.
\end{itemize}
Créez ensuite le fichier archive tar pour cet exercice uniquement en adaptant la structure et les fichiers latex fournis.\\
Montrez le fichier tar à votre professeur.
\newpage
