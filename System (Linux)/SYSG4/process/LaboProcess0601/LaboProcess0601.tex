\lstset{language=c}
\renewcommand{\titre}{\textcolor{blue}{ Process - LaboProcess 06-01 - trap du ctrl-c }}

\lhead{ \titre }
\section{{\titre} }

\begin{tabular}{|l|l|}
\hline
Titre : 	& \titre \\\hline
Support : 	& OS 42.3 Leap Installation Classique \\\hline
Date :		& 07/2011 \\\hline
\end{tabular}

\subsection{Énoncé}

Écrivez un programme qui affiche "aie 1 fois", "aie 2 fois",... "aie n fois", "j'ai compris" lorsqu'on pousse sur ctrl-c. Le nombre n est donné comme premier argument (de 0 à n) et vaut 0 par défaut.

\subsection{Une solution}

\lstinputlisting{LaboProcess0601/SOURCES/Sigaction.c}

\subsection{Commentaires}

\begin{itemize}
\item Pour obtenir de l'aide sur l'utilisation de l'appel système signal : man 2 sigaction
\item Pour obtenir la liste des signaux : man 7 signal
\item Remarquer la boucle infinie dans le programme principal pour laisser à l'utilisateur le temps d'entrer Ctrl-c.
\end{itemize}
\subsection{En roue libre}
Le traitement du signal sera fait une seule fois. On se passera de compteur. 
\newpage
