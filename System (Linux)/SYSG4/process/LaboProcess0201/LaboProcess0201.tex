\lstset{language=c}
\renewcommand{\titre}{\textcolor{blue}{ Process - LaboProcess 02-01 - wait4 }}

\lhead{ \titre }
\section{{\titre} }

\begin{tabular}{|l|l|}
\hline
Titre : 	& \titre \\\hline
Support : 	& OS 42.3 Leap Installation Classique \\\hline
Date :		& 07/2011 \\\hline
\end{tabular}

\subsection{Énoncé}

Écrivez un programme Wait4 qui crée un process fils. \\
Si l'argument est u, ce programme crée un process fils qui effectue le calcul de $10^6$ sinus de nombre aléatoires.\\
Si l'argument est s, ce programme crée un process fils qui effectue l'impression de $10^6$ caractères a\\
Le père attend la fin de son fils et affiche les ressources consommées par son fils.

\subsection{Une solution}

\lstinputlisting{LaboProcess0201/SOURCES/Wait4.c}

\subsection{Commentaires}

\begin{itemize}
\item L'argument u (pour utilisateur) demande beaucoup de ressources utilisateur sans faire appel au S.E 
\item L'argument s (pour système) demande beaucoup de ressources utilisateur et fait appel au S.E avec l'appel système write.
\item On a choisi ici d'écrire dans un périphérique /dev/null qui reçoit les informations mais ne les utilise pas. Ceci évite un affichage de a à l'écran qui prend beaucoup de temps.
\item La structure rusage peut être lue via "man getrusage".
\item Le préfixe time permet de demander au shell la consommation d'une commande (man time).
\end{itemize}
\newpage
