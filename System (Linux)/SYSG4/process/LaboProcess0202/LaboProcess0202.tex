\lstset{language={}}
\renewcommand{\titre}{\textcolor{blue}{ Process - LaboProcess 02-02 - zombies }}

\lhead{ \titre }
\section{{\titre} }

\begin{tabular}{|l|l|}
\hline
Titre : 	& \titre \\\hline
Support : 	& OS 42.3 Leap Installation Classique \\\hline
Date :		& 02/2016 \\\hline
\end{tabular}

\subsection{Énoncé}

Écrivez un programme qui crée un zombie, affiche ensuite cet état zombie à l'aide de la fonction system, élimine ce zombie et affiche les process en cours (sans zombie).

\subsection{Une solution}

\lstinputlisting{LaboProcess0202/SOURCES/Zombie.c}

\subsection{Commentaires}

\begin{itemize}
\item sprintf est la façon propre de transformer un nombre en chaîne de caractère. Écrivez un printf normal, ensuite ajoutez la chaîne destination pré allouée comme premier argument de sprintf.
\item La fonction c system crée un shell et exécute la chaîne comme seule commande de ce shell.
\item l'élimination du zombie est provoquée par l'appel système wait. Un kill n'a pas d'effet sur un Zombie, on ne tue pas un fantôme !
\end{itemize}

\subsection{En roue libre}
Adaptez ce code pour éliminer le zombie grâce à la technique du double fork qui permet de forcer l'adoption immédiate par le process 1\\
Vérifiez qu'il n'y a pas de création de zombies dans ce cas.
\newpage
