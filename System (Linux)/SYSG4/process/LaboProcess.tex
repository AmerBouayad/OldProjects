\documentclass[french,10pt,A4]{report}
%\usepackage[latin1]{inputenc}
\usepackage[utf8]{inputenc}
\usepackage[francais]{babel}
% Modification des marges ------------------------------
\oddsidemargin -4mm % Decreases the left margin by 4mm
\textwidth 17cm %Sets text width across page = 17cm
\textheight 22cm %Sets text height up and down = 22cm
% -----------------------------------------------------
\usepackage[T1]{fontenc}
\usepackage{graphicx,color, caption2}
\usepackage{epsfig}
\usepackage{fancyhdr}
\pagestyle{fancy}
\usepackage{listings}
\usepackage{color}
\usepackage{makeidx}

% Valeurs par défaut le lstset
\lstset{language={},%C,Assembleur, TeX, tcl, basic, cobol, fortran, logo, make, pascal, perl, prolog, {}
	literate={â}{{\^a}}1 {ê}{{\^e}}1 {î}{{\^i}}1 {ô}{{\^o}}1 {û}{{\^u}}1
		 {ä}{{\"a}}1 {ë}{{\"e}}1 {ï}{{\"i}}1 {ö}{{\"o}}1 {ü}{{\"u}}1
		 {à}{{\`a}}1 {é}{{\'e}}1 {è}{{\`e}}1 {ù}{{\`u}}1 
		 {Â}{{\^A}}1 {Ê}{{\^E}}1 {Î}{{\^I}}1 {Ô}{{\^O}}1 {Û}{{\^U}}1
		 {Ä}{{\"A}}1 {Ë}{{\"E}}1 {Ï}{{\"I}}1 {Ö}{{\"O}}1 {Ü}{{\"U}}1
		 {À}{{\`A}}1 {É}{{\'E}}1 {È}{{\`E}}1 {Ù}{{\`U}}1
		 {ç}{{\c c}}1,
   	commentstyle=\scriptsize\ttfamily\slshape, % style des commentaires
	basicstyle=\scriptsize\ttfamily, % style par défaut
   	keywordstyle=\scriptsize\rmfamily\bfseries,% style des mots-clés
   	backgroundcolor=\color[rgb]{.95,.95,.95}, % couleur de fond : gris clair
   	framerule=0.5pt,% Taille des bords
   	frame=trbl,% Style du cadre
 	frameround=tttt, % Bords arrondis 
   	tabsize=3, % Taille des tabulations
%	extendedchars=\true, % Incompatible avec utf8 et literate
	inputencoding=utf8, 
 	showspaces=false, % Ne montre pas les espaces 
 	showstringspaces=false, % Ne montre pas les espaces entre ''
 	xrightmargin=-1cm, % Retrait gauche 
 	xleftmargin=-1cm, % Retrait droit
   	escapechar=°}  % Caractère d'échappement, permet des commandes latex dans la source
% -----------------------------------------------------
%\makeindex
\begin{document}
\lhead{Labo F.S. S.E 2$^eme$}
\rhead{Page \thepage}
\lfoot{\copyright J.C. Jaumain }
\rfoot{\today}
\cfoot{ }
\renewcommand{\footrulewidth}{0.4pt}

\setlength{\parindent}{0pt} % pas d'indentation

\lstset{frame=trBL}

\setcounter{tocdepth}{1}	% limiter les nivaux de table des matières
\setcounter{secnumdepth}{5}	% La numérotation des sections au maximum

\newcommand{\titre}{Titre du sujet}	% la variable contenant le titre du sujet

\thispagestyle{empty}

\title{\emph{Laboratoire\\\textbf{Process}}}
\author{Jaumain J-C}
\date{révision mba - octobre 2018}
\maketitle
\tableofcontents
% ~\\[5cm]
% \flushright{Comment faire réfléchir un imbécile ?}
% ~\\[1cm]
% En lui disant 452 !
% \flushleft{ }

%Ce n'est pas en continuant de faire ce que l'on connait
%que l'on pourra faire ce que l'on ne connait pas

\newpage
%
\chapter{fork}
	\lstset{language=c}
\renewcommand{\titre}{\textcolor{blue}{ Process - LaboProcess 01-01 - fork  clonage et adoption }}

\lhead{ \titre }
\section{{\titre} }

\begin{tabular}{|l|l|}
\hline
Titre : 	& \titre \\\hline
Support : 	& OS 42.3 Leap Installation Classique \\\hline
Date :		& 07/2011 \\\hline
\end{tabular}

\subsection{Énoncé}

Écrivez un programme qui se clone et ensuite affiche son pid, le contenu et l'adresse d'une variable locale. Vérifiez l'ordre d'exécution des process, le contenu et l'adresse de la variable. Inversez l'ordre des affichages en utilisant la fonction sleep. Ce résultat est-il déterministe ? Autrement dit, obtiendra-t-on toujours ces mêmes valeurs ?

\subsection{Une solution}

\lstinputlisting{LaboProcess0101/SOURCES/Fork.c}

\subsection{Commentaires}

\begin{itemize}
\item Une fois le process fils crée, les deux process sont indépendants. 
\item Chaque process a son propre espace d'adressage, il n'y a pas de partage de la variable locale.
\item L'adresse de la variable est une adresse relative à l'espace d'adresseage de chaque processus, les espaces sont séparés physiquement.
\item Le shell attend que son fils soit terminé avant d'envoyer le prompt, il n'attend pas son petit-fils. Selon l'ordonnancement, il se peut qu'on ne voit pas l'affichage du prompt, mais ce dernier a bien eu lieu. Pour obtenir un affichage de prompt tardif, on peut ajouter sleep(1) dans le code du fils (cela dépendra du nombre d'autres processus prioritaires dans le système).
\item L'appel système exit est indispensable chez le fils, sinon le process fils continue et affiche encore deux lignes de plus. 
\item L'ordre des impressions n'est pas garanti.
\end{itemize}
\subsection{En roue libre}
\begin{itemize}
\item Vérifiez le comportement pour les variables de classe d'allocation statique et dynamique.
\item Adaptez cet exemple, en illustrant l'adoption par init.
\end{itemize}
\newpage
 % fork, clonage et adoption
\chapter{wait}
	\lstset{language=c}
\renewcommand{\titre}{\textcolor{blue}{ Process - LaboProcess 02-01 - wait4 }}

\lhead{ \titre }
\section{{\titre} }

\begin{tabular}{|l|l|}
\hline
Titre : 	& \titre \\\hline
Support : 	& OS 42.3 Leap Installation Classique \\\hline
Date :		& 07/2011 \\\hline
\end{tabular}

\subsection{Énoncé}

Écrivez un programme Wait4 qui crée un process fils. \\
Si l'argument est u, ce programme crée un process fils qui effectue le calcul de $10^6$ sinus de nombre aléatoires.\\
Si l'argument est s, ce programme crée un process fils qui effectue l'impression de $10^6$ caractères a\\
Le père attend la fin de son fils et affiche les ressources consommées par son fils.

\subsection{Une solution}

\lstinputlisting{LaboProcess0201/SOURCES/Wait4.c}

\subsection{Commentaires}

\begin{itemize}
\item L'argument u (pour utilisateur) demande beaucoup de ressources utilisateur sans faire appel au S.E 
\item L'argument s (pour système) demande beaucoup de ressources utilisateur et fait appel au S.E avec l'appel système write.
\item On a choisi ici d'écrire dans un périphérique /dev/null qui reçoit les informations mais ne les utilise pas. Ceci évite un affichage de a à l'écran qui prend beaucoup de temps.
\item La structure rusage peut être lue via "man getrusage".
\item Le préfixe time permet de demander au shell la consommation d'une commande (man time).
\end{itemize}
\newpage
 % wait4
	\lstset{language={}}
\renewcommand{\titre}{\textcolor{blue}{ Process - LaboProcess 02-02 - zombies }}

\lhead{ \titre }
\section{{\titre} }

\begin{tabular}{|l|l|}
\hline
Titre : 	& \titre \\\hline
Support : 	& OS 42.3 Leap Installation Classique \\\hline
Date :		& 02/2016 \\\hline
\end{tabular}

\subsection{Énoncé}

Écrivez un programme qui crée un zombie, affiche ensuite cet état zombie à l'aide de la fonction system, élimine ce zombie et affiche les process en cours (sans zombie).

\subsection{Une solution}

\lstinputlisting{LaboProcess0202/SOURCES/Zombie.c}

\subsection{Commentaires}

\begin{itemize}
\item sprintf est la façon propre de transformer un nombre en chaîne de caractère. Écrivez un printf normal, ensuite ajoutez la chaîne destination pré allouée comme premier argument de sprintf.
\item La fonction c system crée un shell et exécute la chaîne comme seule commande de ce shell.
\item l'élimination du zombie est provoquée par l'appel système wait. Un kill n'a pas d'effet sur un Zombie, on ne tue pas un fantôme !
\end{itemize}

\subsection{En roue libre}
Adaptez ce code pour éliminer le zombie grâce à la technique du double fork qui permet de forcer l'adoption immédiate par le process 1\\
Vérifiez qu'il n'y a pas de création de zombies dans ce cas.
\newpage
 % zombies
\chapter{exec}
	\lstset{language=c}
\renewcommand{\titre}{\textcolor{blue}{ Process - LaboProcess 03-01 - exec }}

\lhead{ \titre }
\section{{\titre} }

\begin{tabular}{|l|l|}
\hline
Titre : 	& \titre \\\hline
Support : 	& OS 42.3 Leap Installation Classique \\\hline
Date :		& 02/2015 \\\hline
\end{tabular}

\subsection{Énoncé}

Écrivez un programme en C qui réalise la même chose que \\
\fbox{gcc aff.c -o aff \&\& ./aff Message}\\
Le programme aff affiche le premier argument passé (Message)\\
Si aff.c contient une erreur de compilation, un message d'erreur est affiché. 

\subsection{Une solution}

\lstinputlisting{LaboProcess0301/SOURCES/cg.c}
\lstinputlisting{LaboProcess0301/SOURCES/aff.c}

\subsection{Commentaires}

\begin{itemize}
\item exec est le loader qui permet de charger un programme exécutable en mémoire.
\item exec REMPLACE le process courant par le programme exécutable passé comme premier argument. Le process courant est entièrement détruit. Ceci justifie le fork et l'absence de test après exec gcc pour afficher le message d'erreur.
\item gcc, comme toute commande, se termine par exit(0) si il n'y a pas d'erreur. C'est une convention.
\item l'opérateur \&\& du bash est un AND. La commande qui suit le \&\& est exécutée ssi la première se termine par un exit(0) qui veut dire "tout s'est bien passé"
\end{itemize}
\newpage
 % loader
	\lstset{language=c}
\renewcommand{\titre}{\textcolor{blue}{ Process - LaboProcess 03-02 - exec et sécurité }}

\lhead{ \titre }
\section{{\titre} }

\begin{tabular}{|l|l|}
\hline
Titre : 	& \titre \\\hline
Support : 	& OS 42.3 Leap Installation Classique \\\hline
Date :		& 02/2015 \\\hline
\end{tabular}

\subsection{Énoncé}

Créez un fichier appelé Confidentiel qui contient le mot 'CONFIDENTIEL'. 
Écrivez un programme Conf qui affiche le contenu du fichier appelé Confidentiel en chargeant l'exécutable de la commande cat. 
Modifiez les droits de Confidentiel et de Conf de telle façon que personne d'autre que vous ne puisse afficher le contenu du fichier appelé Confidentiel via la commande cat, mais que tout le monde puisse afficher le contenu en exécutant le programme Conf.

\subsection{Une solution}

\lstinputlisting{LaboProcess0302/SOURCES/Conf.c}

\subsection{Commentaires}

\begin{list}{.}{}
\item Cette solution est dangereuse. exec et SUID ne vont pas bien ensemble
\item Un utilisateur user1 peut se faire passer pour l'utilisateur propriétaire pour d'autres commandes que cat en utilisant le programme Conf.
\end{list}
%Renommez votre Makefile et faites un clean
\newpage
 % exec..p + SUID
\chapter{shell}
	\lstset{language=c}
\renewcommand{\titre}{\textcolor{blue}{ Process - LaboProcess 04-01 - shell simple }}

\lhead{ \titre }
\section{{\titre} }

\begin{tabular}{|l|l|}
\hline
Titre : 	& \titre \\\hline
Support : 	& OS 42.3 Leap Installation Classique \\\hline
Date :		& 07/2011 \\\hline
\end{tabular}

\subsection{Énoncé}

Écrire un shell simple capable d'exécuter n'importe quelle commande externe avec un nombre d'argument quelconque, et la commande interne exit. 

\subsection{Une solution}

\lstinputlisting{LaboProcess0401/SOURCES/Shell.c}

\subsection{Commentaires}

\begin{itemize}
\item Le message 'commande invalide' ne demande pas de if. Si exec a trouvé l'exécutable, il écrase le process courant.
\item Le fork clone le shell courant en un deuxième shell. Ce fils est écrasé par un exécutable si exec le trouve. Sinon, le fils reste un shell. Si on oublie de terminer le fils avec exit(), celui-ci reste un shell et deux shell s'exécutent. L'utilisateur devra alors taper deux fois la commande interne exit pour en sortir.
%\item Le code en commentaire permet d'utiliser ce shell avec un pipe, \fbox{echo ls | ./Shell} par exemple.
\item Le script Demo contient le symbole <<. Ce n'est pas une indirection mais définit le symbole de fin de donnée pour le process.
\end{itemize}
\subsection{En roue libre}
\begin{itemize}
\item Corrigez ce shell simple : votre shell doit se comporter correctement si l'utilisateur introduit le nom d'un exécutable qui n'existe pas.
\end{itemize}
\newpage
 % shell
	\lstset{language=c}
\renewcommand{\titre}{\textcolor{blue}{ Process - LaboProcess 04-02 - shell et background }}

\lhead{ \titre }
\section{{\titre} }

\begin{tabular}{|l|l|}
\hline
Titre : 	& \titre \\\hline
Support : 	& OS 42.3 Leap Installation Classique \\\hline
Date :		& 02/2015 \\\hline
\end{tabular}

\subsection{Énoncé}

Modifier le shell simple pour y ajouter la notion de background.

\subsection{Une solution}

\lstinputlisting{LaboProcess0402/SOURCES/ShellBack.c}

\subsection{Commentaires}

\begin{itemize}
\item bg est mis à 0 à chaque passage et vaut 1 si le dernier mot de la ligne est \&  
\item Le test i>0 empêche que la lecture de tokens[i-1] ne provoque une erreur de segmentation si la ligne est vide.
\item tokens[i-1] est remis à 0 pour que le symbole \& ne soit pas passé à la commande.
\item Un process en bg => pas de wait => le process reste zombie tant que le shell ne se termine pas. Ce problème sera réglé lors des l'étude du chapitre suivant (IPC, les signaux).
\item waitpid et pas wait qui est insuffisant ici. Si vous utilisez un simple wait :
	\begin{itemize}
	\item Vous lancez un process court en bg;
	\item Le process en bg se termine, vous gardez ce process fils à l'état zombie;
	\item Vous lancez une commande longue en fg;
	\item wait (0) attend la fin d'un fils quelconque, le zombie dans ce cas puisqu'il est fini. 
	\item Le zombie est libéré, le père débloqué affiche le prompt
	\item On aurait une inversion de comportement : le process en fg s'exécute comme si il était en bg.
	\end{itemize}
\end{itemize}
\newpage
 % shell et background
	\lstset{language=c}
\renewcommand{\titre}{\textcolor{blue}{ Process - LaboProcess 04-03 - shell et redirections }}

\lhead{ \titre }
\section{{\titre} }

\begin{tabular}{|l|l|}
\hline
Titre : 	& \titre \\\hline
Support : 	& OS 42.3 Leap Installation Classique \\\hline
Date :		& 07/2011 \\\hline
\end{tabular}

\subsection{Énoncé}

Modifier le shell simple pour y ajouter la notion de redirection >.

\subsection{Une solution}

\lstinputlisting{LaboProcess0403/SOURCES/ShellRedir.c}

\subsection{Commentaires}

\begin{itemize}
\item open doit être appelé pour créer le fichier 'fichier'. Il ne faut pas oublier d'ajouter les droits sinon l'appel système prend le contenu de EDX pour les donner ce qui donne n'importe quoi !
\item 0644, le 0 devant signifie 'en octal'.
\end{itemize}

\subsection{En roue libre}
Adaptez cette solution pour la redirection de l'entrée standard "<

\newpage
 % shell et redirections
\chapter{pipe}
	\lstset{language=c}
\renewcommand{\titre}{\textcolor{blue}{ Process - LaboProcess 05-01 - pipe et shell }}

\lhead{ \titre }
\section{{\titre} }

\begin{tabular}{|l|l|}
\hline
Titre : 	& \titre \\\hline
Support : 	& OS 42.3 Leap Installation Classique \\\hline
Date :		& 07/2011 \\\hline
\end{tabular}

\subsection{Énoncé}

Écrivez un shell simplifié, seulement capable d'exécuter toujours la commande \verb+ps aux | grep root | wc -l+ ou la commande exit pour terminer. Ce shell ne reconnaît que la commande exit.

\subsection{Une solution}

\lstinputlisting{LaboProcess0501/SOURCES/PipeShell.c}

\subsection{Commentaires}

\begin{itemize}
\item Une ligne avec 3 commandes
\item Le shell, 3 fork() et 2 pipe() = 16 close !
\item N'oublier aucun close, y compris chez le père.
\item Pas de wait entre les fork() : ces process doivent s'exécuter 'en même temps'.
\item Les close() avant le wait() chez le père sinon deadlock !
\end{itemize}
\newpage
 % pipe complexe
	\lstset{language=c}
\renewcommand{\titre}{\textcolor{blue}{ Process - LaboProcess 05-02 - pipe et shell - à corriger}}

\lhead{ \titre }
\section{{\titre} }

\begin{tabular}{|l|l|}
\hline
Titre : 	& \titre \\\hline
Support : 	& OS 42.3 Leap Installation Classique \\\hline
Date :		& 07/2011 \\\hline
\end{tabular}

\subsection{Énoncé}

Ce programme est erroné. Il devrait exécuter la commande \verb+cat /etc/passwd | cut -f6 -d ':'| sort+ ou la commande exit pour terminer.\\
Vérifiez-en le comportement et expliquez le. Terminez en corrigeant les erreurs.

\subsection{Une solution}

\lstinputlisting{LaboProcess0502/SOURCES/PipeShellErr.c}

\subsection{Commentaires}

\begin{itemize}
\item Une ligne avec 3 commandes et 2 pipes --> 3 fork() et 2 pipe().
\item Pas de wait entre les fork() : ces process doivent s'exécuter 'en même temps'.
\item N'oublier aucun close, y compris chez le père.
\item Les close() avant le wait() chez le père sinon deadlock !
\end{itemize}
\newpage
 % pipe & shell
\chapter{signal}
	\lstset{language=c}
\renewcommand{\titre}{\textcolor{blue}{ Process - LaboProcess 06-01 - trap du ctrl-c }}

\lhead{ \titre }
\section{{\titre} }

\begin{tabular}{|l|l|}
\hline
Titre : 	& \titre \\\hline
Support : 	& OS 42.3 Leap Installation Classique \\\hline
Date :		& 07/2011 \\\hline
\end{tabular}

\subsection{Énoncé}

Écrivez un programme qui affiche "aie 1 fois", "aie 2 fois",... "aie n fois", "j'ai compris" lorsqu'on pousse sur ctrl-c. Le nombre n est donné comme premier argument (de 0 à n) et vaut 0 par défaut.

\subsection{Une solution}

\lstinputlisting{LaboProcess0601/SOURCES/Sigaction.c}

\subsection{Commentaires}

\begin{itemize}
\item Pour obtenir de l'aide sur l'utilisation de l'appel système signal : man 2 sigaction
\item Pour obtenir la liste des signaux : man 7 signal
\item Remarquer la boucle infinie dans le programme principal pour laisser à l'utilisateur le temps d'entrer Ctrl-c.
\end{itemize}
\subsection{En roue libre}
Le traitement du signal sera fait une seule fois. On se passera de compteur. 
\newpage
 %
	\lstset{language=c}
\renewcommand{\titre}{\textcolor{blue}{ Process - LaboProcess 06-02 - trap de tous les signaux }}

\lhead{ \titre }
\section{{\titre} }

\begin{tabular}{|l|l|}
\hline
Titre : 	& \titre \\\hline
Support : 	& OS 42.3 Leap Installation Classique \\\hline
Date :		& 07/2011 \\\hline
\end{tabular}

\subsection{Énoncé}
Écrivez un programme qui traite tous les signaux non temps réel. 
Observez le comportement de ce programme en lui envoyant plusieurs signaux et CTRL-C.\\ 
\\

\subsection{Une solution}

\lstinputlisting{LaboProcess0602/SOURCES/Allsig.c}

\subsection{Commentaires}

\begin{itemize}
\item Les signaux SIGKILL et SIGSTP ne peuvent être traités.
\item L'utilisation de l'appel système pause permet d'éviter l'attente active.
\end{itemize}

\subsection{En roue libre}
\begin{itemize}
\item Modifiez Le comportement du programme de manière à ce qu'un signal traité une fois soit ensuite ignoré en utilisant SIG\_IGN.
\end{itemize}
\newpage
 % 

\chapter{Exercices}
	\lhead{ Labo Process - Exercices }

\begin{list}{+}{}


\item Process002: 
Déterminer les valeurs de glob et loc affichées par ce programme. Justifiez votre réponse.
\begin{lstlisting}[frame=trBL]{}
int glob=45;
int main()
{ int loc=78; printf("< %d %d\n",glob,loc);
   if (fork()= =0) { glob=65; loc=32; exit(0);}
   wait(0);
   printf("> %d %d\n",glob,loc);
   exit(0);
}
\end{lstlisting}

\item Process004: 
fork( ) dédouble-t'il la mémoire pointée par un pointeur ou uniquement le pointeur ? Imaginez un exemple qui prouve votre réponse. La valeur du pointeur est-elle la même chez le fils et le père ? Justifiez votre réponse.

\item Process019: 
Écrivez un shell simplifié seulement capable d'exécuter toujours la commande
\fbox{ls -ail >x  2>x} ou \fbox{exit}. Peu importe ce que vous écrirez comme commandes au clavier. 

\item Process025: 
On vous donne un shell simple. Modifiez-le afin d'y ajouter l'interprétation du caractère * (isolé). 

\item Process027: 
On vous donne un shell simple en annexe. Comment réagit ce shell si vous tapez une commande qui n'existe pas ? Expliquez ce comportement et apportez une solution.

\item Process036: 
Vous trouvez un programme suid root rwsr-xr-x qui contient la ligne \\
\fbox{execlp("ps","ps","aux",0)}\\
Décrivez comment vous pouvez devenir administrateur du système sans en connaître le passwd.
Effectuez une démonstration de cette description.

\item Process045: 
Écrivez un programme qui gère les zombies. A chaque boucle, ce programme lit un message au clavier. Si ce message vaut
\begin{itemize} 
\item c  : le programme crée 2 nouveaux zombies et affiche la liste des zombies 'actuels' via la commande ps.
\item d nnnnn : le programme élimine le zombie de pid nnnnn et affiche la liste des zombies 'actuels' via la commande ps.
\item q : le programme s'arrête.
\end{itemize} 
Que deviennent les éventuels zombies restants quand vous quittez ce programme ? 

\item Process049: 
Écrivez un programme qui simule un shell seulement capable d'exécuter toujours la commande
\fbox{cd;ls>>out} ou \fbox{exit}. Peu importe ce que vous écrirez comme commandes au clavier.

\item Process053:
Écrivez shell simplifié seulement capable d'exécuter toujours la commande
\fbox{cd; (cd ..;ls); ls} ou \fbox{exit}. Peu importe ce que vous écrirez comme commandes au clavier. Veillez à programmer le ; et les parenthèses.
 
\item Process055:
Écrivez un shell simplifié seulement capable d'exécuter toujours la commande
\fbox{mkdir brol \&\& (cd brol;>f)} ou \fbox{exit}. Peu importe ce que vous écrirez comme commandes au clavier. Veillez à programmer le ; et les parenthèses.
 
\item Process2\_004: 
Écrivez un shell simplifié seulement capable d'exécuter toujours la commande
\fbox{ls -ail | cat > x} ou \fbox{exit}. Peu importe ce que vous écrirez comme commandes au clavier.

\item Process2\_009: 
Réécrivez la commande sleep en c et baptisez-le Msleep. Elle utilisera les signaux (SIGALRM) et les appels alarm(n) qui envoie le signal SIGALRM après n secondes et pause() qui bloque le processus jusqu'à l'arrivée d'un signal.

\item Process2\_012: 
Écrire un programme c qui permet d'afficher de façon continue le dernier caractère introduit au clavier. (l'appel read() est bloquant!). Songez à communiquer avec un processus fils via un pipe et un signal.

\item Process2\_025 :
Testez le comportement d'un programme qui intercepte les signaux et affiche un message en clair. Par exemple : signal SIGSEGV reçu du au fait que le programme tente d'accéder à des données d'un pointeur non initialisé. Faites de même avec au moins 3 signaux différents.

\item Process2\_027 :
Écrivez un process qui envoie une série de signaux à un autre process. Ces signaux sont différents et envoyés rapidement, dans un ordre quelconque. Montrez que le process qui reçoit les signaux ne les reçoit pas nécessairement dans le même ordre. Expliquez.

\item Process2\_039 :
Écrivez un programme qui affiche 10 fois "bonjour n" où n va de 0 à 9, à distance de 3 secondes. Votre programme n'utilisera pas l'appel système sleep. 

\item Process2\_054(*) :
Écrivez un shell simplifié seulement capable d'exécuter toujours les commandes
\fbox{ls f 2>err | cat; cat err} ou \fbox{exit}. Peu importe ce que vous écrirez comme commandes au clavier.

\end{list}


\end{document}
