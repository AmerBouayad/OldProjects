\lstset{language=c}
\renewcommand{\titre}{\textcolor{blue}{ Process - LaboProcess 05-01 - pipe et shell }}

\lhead{ \titre }
\section{{\titre} }

\begin{tabular}{|l|l|}
\hline
Titre : 	& \titre \\\hline
Support : 	& OS 42.3 Leap Installation Classique \\\hline
Date :		& 07/2011 \\\hline
\end{tabular}

\subsection{Énoncé}

Écrivez un shell simplifié, seulement capable d'exécuter toujours la commande \verb+ps aux | grep root | wc -l+ ou la commande exit pour terminer. Ce shell ne reconnaît que la commande exit.

\subsection{Une solution}

\lstinputlisting{LaboProcess0501/SOURCES/PipeShell.c}

\subsection{Commentaires}

\begin{itemize}
\item Une ligne avec 3 commandes
\item Le shell, 3 fork() et 2 pipe() = 16 close !
\item N'oublier aucun close, y compris chez le père.
\item Pas de wait entre les fork() : ces process doivent s'exécuter 'en même temps'.
\item Les close() avant le wait() chez le père sinon deadlock !
\end{itemize}
\newpage
