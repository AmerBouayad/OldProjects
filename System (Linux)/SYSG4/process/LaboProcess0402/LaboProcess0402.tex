\lstset{language=c}
\renewcommand{\titre}{\textcolor{blue}{ Process - LaboProcess 04-02 - shell et background }}

\lhead{ \titre }
\section{{\titre} }

\begin{tabular}{|l|l|}
\hline
Titre : 	& \titre \\\hline
Support : 	& OS 42.3 Leap Installation Classique \\\hline
Date :		& 02/2015 \\\hline
\end{tabular}

\subsection{Énoncé}

Modifier le shell simple pour y ajouter la notion de background.

\subsection{Une solution}

\lstinputlisting{LaboProcess0402/SOURCES/ShellBack.c}

\subsection{Commentaires}

\begin{itemize}
\item bg est mis à 0 à chaque passage et vaut 1 si le dernier mot de la ligne est \&  
\item Le test i>0 empêche que la lecture de tokens[i-1] ne provoque une erreur de segmentation si la ligne est vide.
\item tokens[i-1] est remis à 0 pour que le symbole \& ne soit pas passé à la commande.
\item Un process en bg => pas de wait => le process reste zombie tant que le shell ne se termine pas. Ce problème sera réglé lors des l'étude du chapitre suivant (IPC, les signaux).
\item waitpid et pas wait qui est insuffisant ici. Si vous utilisez un simple wait :
	\begin{itemize}
	\item Vous lancez un process court en bg;
	\item Le process en bg se termine, vous gardez ce process fils à l'état zombie;
	\item Vous lancez une commande longue en fg;
	\item wait (0) attend la fin d'un fils quelconque, le zombie dans ce cas puisqu'il est fini. 
	\item Le zombie est libéré, le père débloqué affiche le prompt
	\item On aurait une inversion de comportement : le process en fg s'exécute comme si il était en bg.
	\end{itemize}
\end{itemize}
\newpage
