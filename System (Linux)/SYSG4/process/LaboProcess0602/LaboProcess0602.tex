\lstset{language=c}
\renewcommand{\titre}{\textcolor{blue}{ Process - LaboProcess 06-02 - trap de tous les signaux }}

\lhead{ \titre }
\section{{\titre} }

\begin{tabular}{|l|l|}
\hline
Titre : 	& \titre \\\hline
Support : 	& OS 42.3 Leap Installation Classique \\\hline
Date :		& 07/2011 \\\hline
\end{tabular}

\subsection{Énoncé}
Écrivez un programme qui traite tous les signaux non temps réel. 
Observez le comportement de ce programme en lui envoyant plusieurs signaux et CTRL-C.\\ 
\\

\subsection{Une solution}

\lstinputlisting{LaboProcess0602/SOURCES/Allsig.c}

\subsection{Commentaires}

\begin{itemize}
\item Les signaux SIGKILL et SIGSTP ne peuvent être traités.
\item L'utilisation de l'appel système pause permet d'éviter l'attente active.
\end{itemize}

\subsection{En roue libre}
\begin{itemize}
\item Modifiez Le comportement du programme de manière à ce qu'un signal traité une fois soit ensuite ignoré en utilisant SIG\_IGN.
\end{itemize}
\newpage
