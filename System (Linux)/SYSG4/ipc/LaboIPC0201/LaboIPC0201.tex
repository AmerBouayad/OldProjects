\lstset{language=c}
\renewcommand{\titre}{\textcolor{blue}{ IPC - LaboIPC 02-01 - section critique }}

\lhead{ \titre }
\section{{\titre} }

\begin{tabular}{|l|l|}
\hline
Titre : 	& \titre \\\hline
Support : 	& MDV2007 Installation Classique \\\hline
Date :		& 07/2011 \\\hline
\end{tabular}

\subsection{Énoncé}

Deux process, père et fils, affichent à l'écran ce qui est saisi au clavier. Il faut synchroniser les process de telle façon que l'utilisateur sait à quel process il s'adresse et que l'affichage précise quel process affiche. Ce programme utilise un argument qui vaut soit s (les process sont synchronisés), soit ns (les process ne sont pas synchronisés). Prendre s par défaut.

\subsection{Une solution}

\lstinputlisting{LaboIPC0201/SOURCES/Critique.c}

\subsection{Commentaires}

\begin{itemize}
\item Dans le cas d'un père et une fils l'identifiant est transmis  au moment du fork. Nous choisissons d'utiliser IPC\_PRIVATE.
\item La ressource est mise à 1 au début pour écrire la section critique (lecture + affichage) de chaque processus
\item La section critique doit être la plus petite possible sinon ce n'est pas la peine d'avoir la multiprogrammation.
\item Le père utilise l'appel système wait() avant la suppression du sémaphore par IPC\_RMID(). Est-ce indispensable ?
\item Les variables n et buff sont des instances différentes chez le père et chez le fils.
\end{itemize}

\subsection{En roue libre}
Vérifiez si un sémaphore est supprimé après la mort des process qui l'utilisent. \\
\bigskip
En vous inspirant de l'exemple donné, créez un module source c (une librairie) avec les outils suivants :\\
\begin{itemize}
\item int creeSem (); // sans paramètres, crée un nouveau sémaphore unique et en retourne l'id
\item void initsem (int sem, int val); // initialise le compteur de ressources à val 
\item void supsem (int sem); // supprime le sémaphore
\item void down (int sem);   // obtient une ressource
\item void up (int sem);     // restitue une ressource
\item void zero (int sem);   // attend que le compteur vaille 0
\end{itemize}
\bigskip
testez vos fonctions dans l'exercice précédent.
\newpage
