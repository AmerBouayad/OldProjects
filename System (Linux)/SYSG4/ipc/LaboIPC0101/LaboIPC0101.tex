
\renewcommand{\titre}{\textcolor{blue}{IPC - LaboIPC 01-01 - permanence et suppression}}

\lhead{ \titre }
\section{{\titre} }

\begin{tabular}{|l|l|}
\hline
Titre : 	& \titre \\\hline
Support : 	& MDV2007 Installation Classique \\\hline
Date :		& 07/2011 \\\hline
\end{tabular}

\subsection{Énoncé}

Écrire un programme FreeShm.c qui efface toute les mémoires partagées qui existent sur le système après avoir affiché les informations concernant ces mémoires.
Afin de tester ce programme, écrire un programme AlloueShm.c qui réserve et initialise de la mémoire partagée. Ajouter le programme AttacheShm.c qui consulte les zones de mémoire allouées.

\subsection{Une solution}

\lstinputlisting[language=c]{LaboIPC0101/SOURCES/FreeShm.c}
\lstinputlisting[language=c]{LaboIPC0101/SOURCES/AlloueShm.c}
\lstinputlisting[language=c]{LaboIPC0101/SOURCES/AttachShm.c}

\subsection{Commentaires}

\begin{itemize}
\item Les zones de mémoire allouées sont permanentes. Un même identifiant indique qu'il s'agit de la même zone.
\item On ne peut libérer que les mémoires partagées dont on est propriétaire, sauf le root qui a tous les droits.
\end{itemize}

\subsection{En roue libre}
\begin{itemize}
\item Montrez que l'adresse d'attachement de la mémoire partagée à l'espace d'adressage du programme est un multiple de la taille d'une page.
\item Un processus peut-il continuer à modifier la sone de mémoire partagée après sa "suppression" par le programme FreeShm ? Quel comportement vérifie-t-on lors de nouveaux appels à shmget et shmat pour la même clé après marquage pour suppression ?
\item Comment adapter AlloueShm.c et AttachShm.c en imposant l'utilisation de IPC\_PRIVATE ?
\end{itemize}
\newpage
