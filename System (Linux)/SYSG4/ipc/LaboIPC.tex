\documentclass[french,10pt,A4]{report}
%\usepackage[latin1]{inputenc}
\usepackage[utf8]{inputenc}
\usepackage[francais]{babel}
% Modification des marges ------------------------------
\oddsidemargin -4mm % Decreases the left margin by 4mm
\textwidth 17cm %Sets text width across page = 17cm
\textheight 22cm %Sets text height up and down = 22cm
% -----------------------------------------------------
\usepackage[T1]{fontenc}
\usepackage{graphicx,color, caption2}
\usepackage{epsfig}
\usepackage{fancyhdr}
\pagestyle{fancy}
\usepackage{listings}
\usepackage{color}
\usepackage{makeidx}

% Valeurs par défaut le lstset
\lstset{language={},%C,Assembleur, TeX, tcl, basic, cobol, fortran, logo, make, pascal, perl, prolog, {}
	literate={â}{{\^a}}1 {ê}{{\^e}}1 {î}{{\^i}}1 {ô}{{\^o}}1 {û}{{\^u}}1
		 {ä}{{\"a}}1 {ë}{{\"e}}1 {ï}{{\"i}}1 {ö}{{\"o}}1 {ü}{{\"u}}1
		 {à}{{\`a}}1 {é}{{\'e}}1 {è}{{\`e}}1 {ù}{{\`u}}1 
		 {Â}{{\^A}}1 {Ê}{{\^E}}1 {Î}{{\^I}}1 {Ô}{{\^O}}1 {Û}{{\^U}}1
		 {Ä}{{\"A}}1 {Ë}{{\"E}}1 {Ï}{{\"I}}1 {Ö}{{\"O}}1 {Ü}{{\"U}}1
		 {À}{{\`A}}1 {É}{{\'E}}1 {È}{{\`E}}1 {Ù}{{\`U}}1
		 {ç}{{\c c}}1,
   	commentstyle=\scriptsize\ttfamily\slshape, % style des commentaires
	basicstyle=\scriptsize\ttfamily, % style par défaut
   	keywordstyle=\scriptsize\rmfamily\bfseries,% style des mots-clés
   	backgroundcolor=\color[rgb]{.95,.95,.95}, % couleur de fond : gris clair
   	framerule=0.5pt,% Taille des bords
   	frame=trbl,% Style du cadre
 	frameround=tttt, % Bords arrondis 
   	tabsize=3, % Taille des tabulations
%	extendedchars=\true, % Incompatible avec utf8 et literate
	inputencoding=utf8, 
 	showspaces=false, % Ne montre pas les espaces 
 	showstringspaces=false, % Ne montre pas les espaces entre ''
 	xrightmargin=-1cm, % Retrait gauche 
 	xleftmargin=-1cm, % Retrait droit
   	escapechar=°}  % Caractère d'échappement, permet des commandes latex dans la source
% -----------------------------------------------------
%\makeindex
\begin{document}
\lhead{Labo I.P.C. S.E 2$^eme$}
\rhead{Page \thepage}
\lfoot{\copyright J.C. Jaumain }
\rfoot{\today}
\cfoot{ }
\renewcommand{\footrulewidth}{0.4pt}

\setlength{\parindent}{0pt} % pas d'indentation

\lstset{frame=trBL}

\setcounter{tocdepth}{1}	% limiter les nivaux de table des matières
\setcounter{secnumdepth}{5}	% La numérotation des sections au maximum

\newcommand{\titre}{Titre du sujet}	% la variable contenant le titre du sujet

\thispagestyle{empty}

\title{\emph{Laboratoire\\\textbf{I.P.C.}}}
\author{Jaumain J-C}
\date{15 janvier 2014}
\date{révision mba - Novembre 2018}
\maketitle
\tableofcontents
% ~\\[5cm]
% \flushright{}
% - Il n'y a que les imbéciles qui sont absolument certains de quelque chose\\
% - Vous en êtes certain ?\\
% - Absolument !\\
% \flushleft{}
\newpage
%
%\chapter{Pipes}
	%
\renewcommand{\titre}{\textcolor{blue}{IPC - LaboIPC 01-01 - permanence et suppression}}

\lhead{ \titre }
\section{{\titre} }

\begin{tabular}{|l|l|}
\hline
Titre : 	& \titre \\\hline
Support : 	& MDV2007 Installation Classique \\\hline
Date :		& 07/2011 \\\hline
\end{tabular}

\subsection{Énoncé}

Écrire un programme FreeShm.c qui efface toute les mémoires partagées qui existent sur le système après avoir affiché les informations concernant ces mémoires.
Afin de tester ce programme, écrire un programme AlloueShm.c qui réserve et initialise de la mémoire partagée. Ajouter le programme AttacheShm.c qui consulte les zones de mémoire allouées.

\subsection{Une solution}

\lstinputlisting[language=c]{LaboIPC0101/SOURCES/FreeShm.c}
\lstinputlisting[language=c]{LaboIPC0101/SOURCES/AlloueShm.c}
\lstinputlisting[language=c]{LaboIPC0101/SOURCES/AttachShm.c}

\subsection{Commentaires}

\begin{itemize}
\item Les zones de mémoire allouées sont permanentes. Un même identifiant indique qu'il s'agit de la même zone.
\item On ne peut libérer que les mémoires partagées dont on est propriétaire, sauf le root qui a tous les droits.
\end{itemize}

\subsection{En roue libre}
\begin{itemize}
\item Montrez que l'adresse d'attachement de la mémoire partagée à l'espace d'adressage du programme est un multiple de la taille d'une page.
\item Un processus peut-il continuer à modifier la sone de mémoire partagée après sa "suppression" par le programme FreeShm ? Quel comportement vérifie-t-on lors de nouveaux appels à shmget et shmat pour la même clé après marquage pour suppression ?
\item Comment adapter AlloueShm.c et AttachShm.c en imposant l'utilisation de IPC\_PRIVATE ?
\end{itemize}
\newpage
 % pipe et sort
	%\input{LaboIPC0102/LaboIPC0102} % pipe et shell
%\chapter{Signaux}
	%\lstset{language=c}
\renewcommand{\titre}{\textcolor{blue}{ IPC - LaboIPC 02-01 - section critique }}

\lhead{ \titre }
\section{{\titre} }

\begin{tabular}{|l|l|}
\hline
Titre : 	& \titre \\\hline
Support : 	& MDV2007 Installation Classique \\\hline
Date :		& 07/2011 \\\hline
\end{tabular}

\subsection{Énoncé}

Deux process, père et fils, affichent à l'écran ce qui est saisi au clavier. Il faut synchroniser les process de telle façon que l'utilisateur sait à quel process il s'adresse et que l'affichage précise quel process affiche. Ce programme utilise un argument qui vaut soit s (les process sont synchronisés), soit ns (les process ne sont pas synchronisés). Prendre s par défaut.

\subsection{Une solution}

\lstinputlisting{LaboIPC0201/SOURCES/Critique.c}

\subsection{Commentaires}

\begin{itemize}
\item Dans le cas d'un père et une fils l'identifiant est transmis  au moment du fork. Nous choisissons d'utiliser IPC\_PRIVATE.
\item La ressource est mise à 1 au début pour écrire la section critique (lecture + affichage) de chaque processus
\item La section critique doit être la plus petite possible sinon ce n'est pas la peine d'avoir la multiprogrammation.
\item Le père utilise l'appel système wait() avant la suppression du sémaphore par IPC\_RMID(). Est-ce indispensable ?
\item Les variables n et buff sont des instances différentes chez le père et chez le fils.
\end{itemize}

\subsection{En roue libre}
Vérifiez si un sémaphore est supprimé après la mort des process qui l'utilisent. \\
\bigskip
En vous inspirant de l'exemple donné, créez un module source c (une librairie) avec les outils suivants :\\
\begin{itemize}
\item int creeSem (); // sans paramètres, crée un nouveau sémaphore unique et en retourne l'id
\item void initsem (int sem, int val); // initialise le compteur de ressources à val 
\item void supsem (int sem); // supprime le sémaphore
\item void down (int sem);   // obtient une ressource
\item void up (int sem);     // restitue une ressource
\item void zero (int sem);   // attend que le compteur vaille 0
\end{itemize}
\bigskip
testez vos fonctions dans l'exercice précédent.
\newpage
 % trap du ctrl c
	%\lstset{language=c}
\renewcommand{\titre}{\textcolor{blue}{ IPC - LaboIPC 02-02 - synchronisation pour Affichage continu }}

\lhead{ \titre }
\section{{\titre} }

\begin{tabular}{|l|l|}
\hline
Titre : 	& \titre \\\hline
Support : 	& MDV2007 Installation Classique \\\hline
Date :		& 07/2011 \\\hline
\end{tabular}

\subsection{Énoncé}

Écrire un programme AffContinu.c qui affiche de façon permanente un caractère saisi au clavier. L'affichage ne peut commencer qu'après avoir saisi un premier caractère. Exemple, si on saisi 'a' suivi de enter, des 'a' s'affichent de façon continue à l'écran...etc. Le programme s'arrête si on frappe 'q' suivi de enter. L'ordre de lancement est quelconque.

\subsection{Une solution}

\lstinputlisting{LaboIPC0202/SOURCES/AffContinu.c}

\subsection{Commentaires}

\begin{itemize}
\item La lecture est bloquante, l'affichage est continu. Il faut donc deux process.
\item Il faut communiquer le caractère lu à l'autre process. Nous utilisons une mémoire partagée à ce fin
\item Le sémaphore est utilisé pour empêcher le process qui affiche de commencer avant celui qui lit.
\item Au clavier, il faut lire au minimum 2 caractères (celui souhaité + enter). Seul le premier est affiché. 
\item Il est utile de vérifier que tous ses process sont terminés à l'aide de ps.
\end{itemize}

\subsection{En roue libre}
\begin{itemize}
\item Modifiez la logique de synchronisation de départ pour que l'on puisse utiliser un processus afficheur indépendant (compilé séparément). 
\item Comment l'adapter si le nombre d'afficheurs n'est pas connu ?
\end{itemize}
\newpage

 % signaux et exec
	%\lstset{language=c}
\renewcommand{\titre}{\textcolor{blue}{ IPC - LaboIPC 02-03 - synchronisation accès aux douches }}

\lhead{ \titre }
\section{{\titre} }

\begin{tabular}{|l|l|}
\hline
Titre : 	& \titre \\\hline
Support : 	& MDV2007 Installation Classique \\\hline
Date :		& 07/2011 \\\hline
\end{tabular}

\subsection{En roue libre}
\begin{itemize}
\item Gérez une douche où ne peuvent se trouver en même temps filles (processus F) et garçons (processus G). 
\item Un garçon ne pourra entrer en présence d'au moins une fille et vice-versa.
\item Le nombre de filles et garçons n'est pas connu et est quelconque.
\end{itemize}
\newpage
 % signaux et fork
	%
\renewcommand{\titre}{\textcolor{blue}{ IPC - LaboIPC 02-04 - producteur-consommateur}}

\lhead{ \titre }
\section{{\titre} }

\begin{tabular}{|l|l|}
\hline
Titre : 	& \titre \\\hline
Support : 	& MDV2007 Installation Classique \\\hline
Date :		& 11/2017 \\\hline
\end{tabular}

\subsection{Énoncé}

Écrire deux programmes prod.c et cons.c. Ils simulent le problème du producteur-consommateur. Le producteur lit stdin. Le consommateur affiche sur stdout.
La mémoire partagée sera un tableau de 5 caractères.

\subsection{Une solution}

\lstinputlisting[language=c]{LaboIPC0204/SOURCES/cons.c}
\lstinputlisting[language=c]{LaboIPC0204/SOURCES/prod.c}

\subsection{Commentaires}

\begin{itemize}
\item Au cas où ce programme ne donne pas les résultats espérés, commencer par vérifier que les sémaphores sont bien libres.
\item Le sémaphore PLEIN signifie nombre de cases pleines, 0 au début
\item Le sémaphore VIDE signifie nombre de cases libres, 5 au début
\item La fonction opsem est celle définie au laboratoire 0202.
\end{itemize}

\subsection{En roue Libre}

\begin{itemize}
\item Généralisez ce programme au cas où il y aurait plusieurs consommateurs qui tournent en même temps. Il faut partager et protéger l'accès à la variable tete. Vous aurez besoin d'une section critique.
\item La suppression des sémaphores est immédiate, elle provoque une erreur dans le consommateur.
\end{itemize}
\newpage
 % shell et zombies
\chapter{mémoire partagée \& permanence}
	
\renewcommand{\titre}{\textcolor{blue}{IPC - LaboIPC 01-01 - permanence et suppression}}

\lhead{ \titre }
\section{{\titre} }

\begin{tabular}{|l|l|}
\hline
Titre : 	& \titre \\\hline
Support : 	& MDV2007 Installation Classique \\\hline
Date :		& 07/2011 \\\hline
\end{tabular}

\subsection{Énoncé}

Écrire un programme FreeShm.c qui efface toute les mémoires partagées qui existent sur le système après avoir affiché les informations concernant ces mémoires.
Afin de tester ce programme, écrire un programme AlloueShm.c qui réserve et initialise de la mémoire partagée. Ajouter le programme AttacheShm.c qui consulte les zones de mémoire allouées.

\subsection{Une solution}

\lstinputlisting[language=c]{LaboIPC0101/SOURCES/FreeShm.c}
\lstinputlisting[language=c]{LaboIPC0101/SOURCES/AlloueShm.c}
\lstinputlisting[language=c]{LaboIPC0101/SOURCES/AttachShm.c}

\subsection{Commentaires}

\begin{itemize}
\item Les zones de mémoire allouées sont permanentes. Un même identifiant indique qu'il s'agit de la même zone.
\item On ne peut libérer que les mémoires partagées dont on est propriétaire, sauf le root qui a tous les droits.
\end{itemize}

\subsection{En roue libre}
\begin{itemize}
\item Montrez que l'adresse d'attachement de la mémoire partagée à l'espace d'adressage du programme est un multiple de la taille d'une page.
\item Un processus peut-il continuer à modifier la sone de mémoire partagée après sa "suppression" par le programme FreeShm ? Quel comportement vérifie-t-on lors de nouveaux appels à shmget et shmat pour la même clé après marquage pour suppression ?
\item Comment adapter AlloueShm.c et AttachShm.c en imposant l'utilisation de IPC\_PRIVATE ?
\end{itemize}
\newpage
 % récupération
\chapter{mémoire partagée \& sémaphores}
	\lstset{language=c}
\renewcommand{\titre}{\textcolor{blue}{ IPC - LaboIPC 02-01 - section critique }}

\lhead{ \titre }
\section{{\titre} }

\begin{tabular}{|l|l|}
\hline
Titre : 	& \titre \\\hline
Support : 	& MDV2007 Installation Classique \\\hline
Date :		& 07/2011 \\\hline
\end{tabular}

\subsection{Énoncé}

Deux process, père et fils, affichent à l'écran ce qui est saisi au clavier. Il faut synchroniser les process de telle façon que l'utilisateur sait à quel process il s'adresse et que l'affichage précise quel process affiche. Ce programme utilise un argument qui vaut soit s (les process sont synchronisés), soit ns (les process ne sont pas synchronisés). Prendre s par défaut.

\subsection{Une solution}

\lstinputlisting{LaboIPC0201/SOURCES/Critique.c}

\subsection{Commentaires}

\begin{itemize}
\item Dans le cas d'un père et une fils l'identifiant est transmis  au moment du fork. Nous choisissons d'utiliser IPC\_PRIVATE.
\item La ressource est mise à 1 au début pour écrire la section critique (lecture + affichage) de chaque processus
\item La section critique doit être la plus petite possible sinon ce n'est pas la peine d'avoir la multiprogrammation.
\item Le père utilise l'appel système wait() avant la suppression du sémaphore par IPC\_RMID(). Est-ce indispensable ?
\item Les variables n et buff sont des instances différentes chez le père et chez le fils.
\end{itemize}

\subsection{En roue libre}
Vérifiez si un sémaphore est supprimé après la mort des process qui l'utilisent. \\
\bigskip
En vous inspirant de l'exemple donné, créez un module source c (une librairie) avec les outils suivants :\\
\begin{itemize}
\item int creeSem (); // sans paramètres, crée un nouveau sémaphore unique et en retourne l'id
\item void initsem (int sem, int val); // initialise le compteur de ressources à val 
\item void supsem (int sem); // supprime le sémaphore
\item void down (int sem);   // obtient une ressource
\item void up (int sem);     // restitue une ressource
\item void zero (int sem);   // attend que le compteur vaille 0
\end{itemize}
\bigskip
testez vos fonctions dans l'exercice précédent.
\newpage
 % section critique 
	\lstset{language=c}
\renewcommand{\titre}{\textcolor{blue}{ IPC - LaboIPC 02-02 - synchronisation pour Affichage continu }}

\lhead{ \titre }
\section{{\titre} }

\begin{tabular}{|l|l|}
\hline
Titre : 	& \titre \\\hline
Support : 	& MDV2007 Installation Classique \\\hline
Date :		& 07/2011 \\\hline
\end{tabular}

\subsection{Énoncé}

Écrire un programme AffContinu.c qui affiche de façon permanente un caractère saisi au clavier. L'affichage ne peut commencer qu'après avoir saisi un premier caractère. Exemple, si on saisi 'a' suivi de enter, des 'a' s'affichent de façon continue à l'écran...etc. Le programme s'arrête si on frappe 'q' suivi de enter. L'ordre de lancement est quelconque.

\subsection{Une solution}

\lstinputlisting{LaboIPC0202/SOURCES/AffContinu.c}

\subsection{Commentaires}

\begin{itemize}
\item La lecture est bloquante, l'affichage est continu. Il faut donc deux process.
\item Il faut communiquer le caractère lu à l'autre process. Nous utilisons une mémoire partagée à ce fin
\item Le sémaphore est utilisé pour empêcher le process qui affiche de commencer avant celui qui lit.
\item Au clavier, il faut lire au minimum 2 caractères (celui souhaité + enter). Seul le premier est affiché. 
\item Il est utile de vérifier que tous ses process sont terminés à l'aide de ps.
\end{itemize}

\subsection{En roue libre}
\begin{itemize}
\item Modifiez la logique de synchronisation de départ pour que l'on puisse utiliser un processus afficheur indépendant (compilé séparément). 
\item Comment l'adapter si le nombre d'afficheurs n'est pas connu ?
\end{itemize}
\newpage

 % affichage continu
	\lstset{language=c}
\renewcommand{\titre}{\textcolor{blue}{ IPC - LaboIPC 02-03 - synchronisation accès aux douches }}

\lhead{ \titre }
\section{{\titre} }

\begin{tabular}{|l|l|}
\hline
Titre : 	& \titre \\\hline
Support : 	& MDV2007 Installation Classique \\\hline
Date :		& 07/2011 \\\hline
\end{tabular}

\subsection{En roue libre}
\begin{itemize}
\item Gérez une douche où ne peuvent se trouver en même temps filles (processus F) et garçons (processus G). 
\item Un garçon ne pourra entrer en présence d'au moins une fille et vice-versa.
\item Le nombre de filles et garçons n'est pas connu et est quelconque.
\end{itemize}
\newpage
 % douches
	
\renewcommand{\titre}{\textcolor{blue}{ IPC - LaboIPC 02-04 - producteur-consommateur}}

\lhead{ \titre }
\section{{\titre} }

\begin{tabular}{|l|l|}
\hline
Titre : 	& \titre \\\hline
Support : 	& MDV2007 Installation Classique \\\hline
Date :		& 11/2017 \\\hline
\end{tabular}

\subsection{Énoncé}

Écrire deux programmes prod.c et cons.c. Ils simulent le problème du producteur-consommateur. Le producteur lit stdin. Le consommateur affiche sur stdout.
La mémoire partagée sera un tableau de 5 caractères.

\subsection{Une solution}

\lstinputlisting[language=c]{LaboIPC0204/SOURCES/cons.c}
\lstinputlisting[language=c]{LaboIPC0204/SOURCES/prod.c}

\subsection{Commentaires}

\begin{itemize}
\item Au cas où ce programme ne donne pas les résultats espérés, commencer par vérifier que les sémaphores sont bien libres.
\item Le sémaphore PLEIN signifie nombre de cases pleines, 0 au début
\item Le sémaphore VIDE signifie nombre de cases libres, 5 au début
\item La fonction opsem est celle définie au laboratoire 0202.
\end{itemize}

\subsection{En roue Libre}

\begin{itemize}
\item Généralisez ce programme au cas où il y aurait plusieurs consommateurs qui tournent en même temps. Il faut partager et protéger l'accès à la variable tete. Vous aurez besoin d'une section critique.
\item La suppression des sémaphores est immédiate, elle provoque une erreur dans le consommateur.
\end{itemize}
\newpage
 % producteur consommateur 
\chapter{Exercices}
	\lhead{ Labo I.P.C. - Exercices }

\begin{list}{+}{}


\item Ipc014: 
Définissez une section critique qu'un seul process peut accéder à la fois. Écrire un process qui utilise cette section critique. Immédiatement après y être entré, et juste avant d'en sortir, il affiche '1'. Écrire un deuxième process qui utilise la même section critique. De la même façon, il affiche '2'.
Quelle sera la séquence logique des caractères 1 et 2 qui vont s'afficher ?

\item Ipc029 :
Écrivez 3 process. Synchronisez ces process de telle façon que le process 3 ne peut s'exécuter que quand les process 1 et 2 sont terminés. L'ordre de lancement des process est quelconque. Prouvez que votre synchronisation est correcte. Vous devez utiliser les sémaphores SystemV pour résoudre cet exercice.

\item Ipc031 :
Soient deux processus; le premier affichant des caractères A à raison de 1 par seconde, le deuxième affichant des caractères B à raison de 1 par seconde.
Réalisez exactement l'alternance AA B AA B AA B AA B AA B AA B ... par synchronisation des deux processus.

\item Ipc032 :
Un process crée une réserve de N cacahouètes. N est un nombre compris entre 1000 et 2000. 3 processus fils sont des mangeurs de cacahouètes (M à la fois). M est un nombre aléatoire compris entre 100 et 200. Les processus fils affichent sur 3 lignes "je suis le processus <pid> , je vole M cacahuètes, je pars". Les fils meurent quand il ne reste plus suffisament de cacahouètes dans la réserve. Le parent meurt quand les 3 fils sont morts. Les messages ne peuvent pas être mélangés et le nombre de cachuètes disponibles respecté. Vous ne pouvez pas utiliser de mémoire partagée dans cet exercice.
%%

\item Ipc046 :
Soient deux processus qui affichent respectivement les caractères A et B à répétition. Arrangez-vous pour obtenir un affichage alterné des deux lettres.
%%\item Ipc047 :
%%Soient deux processus. Le premier processus lit un caractère au clavier, le deuxième l'affiche à l'écran. A chaque nouvelle lecture au clavier, le caractère est affiché une fois. Vous ne pouvez pas utiliser de mémoire partagée.

\item Ipc059 :
Soient deux processus qui affichent respectivement \fbox{bonjour} et \fbox{le monde} à la réception du signal SIGUSR1. Arrangez-vous pour que la phrase \fbox{bonjour le monde} soit toujours écrite dans le bon ordre.
%%\item Ipc060 :
%%Soient deux processus. Le premier process A execute deux fonctions A1 et A2, le deuxième B, 
%%les deux fonctions B1 et B2. Il faut s'assurer que A1 est exécuté avant B2 et que B1 est exécuté avant A2. Trouvez une solution qui ne provoque jamais d'interblocage.
% Autre problème de synchronisation : le rendez-vous (Le premier process A execute deux 
% fonctions A1 et A2, le deuxième B, les deux fonctions B1 et B2. Il faut s'assurer que A1 est
% exécuté avant B2 et que B1 est exécuté avant A2. Trouvez une solution qui ne provoque jamais
% d'interblocage + autres Systèmes d'exploitation, collection Synthex.
% Afficher ce que le serveur ssh envoie.
% signaux aux process bloqués ? Comment sont-ils choisis par l'ordonnanceur ?
% shell qui affiche "redirection ambigüe" avec les pipes et les redirections



%\item Ipc009: 
%Écrivez un processus qui crée une structure chaînée sous forme de liste au sein d'un tableau situé dans une mémoire partagée. Écrivez un second processus qui relit les informations de la liste chaînée créée par le précédent dans l'ordre du chaînage. La structure chaînée contient deux éléments : un pointeur sur le maillon suivant et un pointeur vers une donnée du processus. Cela fonctionne-t-il ? Pourquoi ? Faites en sorte que cela fonctionne.


%Écrire un programme Affiche.c qui affiche, en boucle, les données contenues dans les mémoires partagées comprises entre 100 et 119 si elles existent. Lancer plusieurs exécutions du programme Travail.c et une de Affiche.c et admirer le résultat 
%Remarques : Les statistiques se trouvent dans /proc/[pid]/stat. La définition de ces informations se trouvent dans man proc. La fonction system() en c exécute une ligne de commande en bash.
\end{list}



\end{document}
