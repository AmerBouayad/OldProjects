
\renewcommand{\titre}{\textcolor{blue}{ IPC - LaboIPC 02-04 - producteur-consommateur}}

\lhead{ \titre }
\section{{\titre} }

\begin{tabular}{|l|l|}
\hline
Titre : 	& \titre \\\hline
Support : 	& MDV2007 Installation Classique \\\hline
Date :		& 11/2017 \\\hline
\end{tabular}

\subsection{Énoncé}

Écrire deux programmes prod.c et cons.c. Ils simulent le problème du producteur-consommateur. Le producteur lit stdin. Le consommateur affiche sur stdout.
La mémoire partagée sera un tableau de 5 caractères.

\subsection{Une solution}

\lstinputlisting[language=c]{LaboIPC0204/SOURCES/cons.c}
\lstinputlisting[language=c]{LaboIPC0204/SOURCES/prod.c}

\subsection{Commentaires}

\begin{itemize}
\item Au cas où ce programme ne donne pas les résultats espérés, commencer par vérifier que les sémaphores sont bien libres.
\item Le sémaphore PLEIN signifie nombre de cases pleines, 0 au début
\item Le sémaphore VIDE signifie nombre de cases libres, 5 au début
\item La fonction opsem est celle définie au laboratoire 0202.
\end{itemize}

\subsection{En roue Libre}

\begin{itemize}
\item Généralisez ce programme au cas où il y aurait plusieurs consommateurs qui tournent en même temps. Il faut partager et protéger l'accès à la variable tete. Vous aurez besoin d'une section critique.
\item La suppression des sémaphores est immédiate, elle provoque une erreur dans le consommateur.
\end{itemize}
\newpage
